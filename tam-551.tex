\documentclass[12pt]{article}

%––––– Packages –––––
\usepackage[margin=1in]{geometry}
\usepackage{amsmath,amssymb,amsthm,amsfonts}
\usepackage{enumitem}
\usepackage{hyperref}
\usepackage{xcolor}
\usepackage{import}
\usepackage{xifthen}
\usepackage{pdfpages}
\usepackage{transparent}
\usepackage{listings}
\usepackage{tikz}
\usepackage{physics}
\usepackage{siunitx}
\usepackage{booktabs}
\usepackage{cancel}
\usepackage{bm}

  \usetikzlibrary{calc,patterns,arrows.meta,decorations.markings}

%––––– Custom macros –––––
\newcommand{\incfig}[1]{%%
    \def\svgwidth{\columnwidth}%%
    \import{./Figures/}{##1.pdf_tex}%%
}

%––––– Theorem styles –––––
\theoremstyle{definition}
\newtheorem{definition}{Definition}
\newtheorem{example}{Example}
\newtheorem{problem}{Problem}
\newtheorem{solution}{Solution}
\newtheorem{remark}{Remark}
\theoremstyle{plain}
\newtheorem{theorem}{Theorem}
\newtheorem{lemma}{Lemma}
\newtheorem{proposition}{Proposition}
\newtheorem{corollary}{Corollary}

%––––– Meta –––––
\title{TAM 551: Solid Mechanics I\thanks{Compiled by Jerich Lee}}
\author{Petros Sofronis}
\date{11/01/2025}

\begin{document}
\maketitle
\tableofcontents
\pagebreak

% Your content starts here.
\section{Mathematical Preliminaries}
\subsection{Indicial Notation and Tensor Operations}

---

\subsubsection{Summation Convention}

Consider the sum:
\[
S = a_1 x_1 + a_2 x_2 + a_3 x_3 = \sum_{i=1}^{3} a_i x_i = a_i x_i
\]
Repeated indices (called \emph{dummy indices}) imply summation from \(1\) to \(3\).  
Indices that appear only once in a term are called \emph{free indices} and denote tensor components.

---

\subsubsection{Example}

A system of linear equations:
\[
\begin{cases}
A_{11}x_1 + A_{12}x_2 + A_{13}x_3 = b_1,\\
A_{21}x_1 + A_{22}x_2 + A_{23}x_3 = b_2,\\
A_{31}x_1 + A_{32}x_2 + A_{33}x_3 = b_3,
\end{cases}
\]
is written compactly as:
\[
A_{ij}x_j = b_i
\]
where \(i\) is free (ranges over the three equations), and \(j\) is dummy (summed over).

---

\subsubsection{Transformation of Components}

If \(O_{ij}\) is a transformation matrix, then:
\[
A'_i = O_{ij} A_j
\]
Here, \(i\) is a free index and \(j\) is dummy.

---

\subsection{Kronecker Delta}

\[
\boxed{
\delta_{ij} =
\begin{cases}
1, & i = j,\\
0, & i \neq j.
\end{cases}
}
\]

Properties:
\[
\delta_{ij} = \delta_{ji}, \quad \delta_{ii} = 3, \quad \delta_{ik}\delta_{kj} = \delta_{ij}.
\]

Useful relations:
\[
a_i = \delta_{ij}a_j, \qquad A_{ik}\delta_{kj} = A_{ij}.
\]
For an orthonormal basis \((\mathbf{e}_1, \mathbf{e}_2, \mathbf{e}_3)\):
\[
\mathbf{e}_i \cdot \mathbf{e}_j = \delta_{ij}, \quad
\mathbf{a} \cdot \mathbf{b} = a_i b_i = |\mathbf{a}||\mathbf{b}|\cos\theta.
\]

---

\subsection{Alternating Symbol (Levi-Civita Tensor)}

\[
\boxed{
\varepsilon_{ijk} =
\begin{cases}
+1, & \text{if } (ijk) \text{ is even (e.g. } 123, 231, 312),\\
-1, & \text{if } (ijk) \text{ is odd (e.g. } 132, 321, 213),\\
0, & \text{if any two indices are equal.}
\end{cases}
}
\]

Properties:
\[
\varepsilon_{ijk} = -\varepsilon_{jik} = -\varepsilon_{ikj}, \quad
\varepsilon_{ijk}\varepsilon_{ijk} = 6.
\]

Compact expression:
\[
\varepsilon_{ijk} = \tfrac{1}{2}(i-j)(j-k)(k-i)
\]

---

\subsubsection{Vector Products}

For unit basis vectors:
\[
\mathbf{e}_i \times \mathbf{e}_j = \varepsilon_{ijk}\mathbf{e}_k.
\]

\paragraph{Cross product:}
\[
(\mathbf{a} \times \mathbf{b})_i = \varepsilon_{ijk} a_j b_k.
\]

\paragraph{Triple product:}
\[
(\mathbf{a} \times \mathbf{b}) \cdot \mathbf{c} = \varepsilon_{ijk} a_i b_j c_k.
\]

---

\subsubsection{Useful Identities}

\[
\varepsilon_{ijk}\varepsilon_{ilm} = 
\delta_{jl}\delta_{km} - \delta_{jm}\delta_{kl}.
\]

\[
(\mathbf{a} \times \mathbf{b}) \times \mathbf{c}
= \varepsilon_{ijk}\varepsilon_{klm}a_j b_l c_m
= (a_i b_j c_j - a_j b_i c_j)
= \mathbf{b}(\mathbf{a} \cdot \mathbf{c}) - \mathbf{a}\]
\subsection{Matrix Theory in Index Notation}

---

\subsubsection{Matrix Representation}

\[
[A] =
\begin{bmatrix}
A_{11} & \dots & A_{1n} \\
\vdots & \ddots & \vdots \\
A_{m1} & \dots & A_{mn}
\end{bmatrix}
\]

For matrix multiplication:
\[
[C] = [A][B] \quad \Rightarrow \quad C_{ij} = A_{ik}B_{kj}.
\]

---

\subsubsection{Transpose and Symmetry}

\[
[A^T]_{ij} = A_{ji}.
\]

\[
\text{Symmetric: } A_{ij} = A_{ji}, \quad
\text{Antisymmetric: } A_{ij} = -A_{ji}.
\]

---

\subsection{Determinant}

\[
\boxed{
\det[A] = \varepsilon_{ijk}A_{1i}A_{2j}A_{3k}
}
\]
(for a \(3\times3\) matrix).

Alternatively:
\[
\det[A] = \sum_{i,j,k=1}^3 \varepsilon_{ijk} A_{pi}A_{qj}A_{rk}
\]
where \((p,q,r)\) are fixed row indices.

---

\subsubsection{Minor and Cofactor}

- **Minor:** \( M_{ij} \) is the determinant of the submatrix obtained by deleting the \(i\)th row and \(j\)th column.
- **Cofactor:** \( C_{ij} = (-1)^{i+j} M_{ij} \).

Example for a \(2\times2\) matrix:
\[
C_{11} = A_{22}, \quad C_{22} = A_{11}, \quad
C_{12} = -A_{21}, \quad C_{21} = -A_{12}.
\]
Hence:
\[
\det[A] = A_{11}A_{22} - A_{12}A_{21}.
\]

---

\subsubsection{Properties of Determinants}

\begin{enumerate}
\item If all elements of any row or column are zero, then \(\det[A]=0.\)
\item Interchanging two rows (or columns) changes the sign of the determinant.
\item \(\det[A] = \det[A^T].\)
\item \(\det[AB] = \det[A]\det[B].\)
\end{enumerate}

---

\subsubsection{Row and Column Expansions}

\[
\det[A] = \varepsilon_{ijk}A_{1i}A_{2j}A_{3k} = \varepsilon_{ijk}A_{i1}A_{j2}A_{k3}.
\]

---

\begin{proof}[Column Expansion]
Using \(\det[A^T] = \det[A]\):
\[
\det[A^T] = \varepsilon_{ijk}A_{ip}A_{jq}A_{kr}
\quad \Rightarrow \quad
\det[A] = \varepsilon_{ijk}A_{pi}A_{qj}A_{rk}.
\]
\end{proof}
---

\begin{proof}[Determinant Multiplication Theorem]
 
Let \( [C] = [A][B] \Rightarrow C_{ij} = A_{ik}B_{kj}. \)

\[
\det[C] = \varepsilon_{ijk}C_{1i}C_{2j}C_{3k}
= \varepsilon_{ijk}A_{1p}A_{2q}A_{3r}B_{pi}B_{qj}B_{rk}.
\]

Using the identity:
\[
\varepsilon_{ijk}B_{pi}B_{qj}B_{rk} = (\det B)\varepsilon_{pqr},
\]
we obtain:
\[
\det[C] = (\det A)(\det B).
\]

 
\end{proof}
\subsubsection{Proof: }
---

\subsection{Properties of Levi-Civita Symbol}

\[
\varepsilon_{ijk}\varepsilon_{ipq} = 
\begin{vmatrix}
\delta_{jp} & \delta_{jq} \\
\delta_{kp} & \delta_{kq}
\end{vmatrix}
= \delta_{jq}\delta_{kp} - \delta_{jp}\delta_{kq}.
\]

\[
\varepsilon_{ijk}\varepsilon_{ijk} = 6, \quad
\varepsilon_{ijk}\varepsilon_{ijp} = 2\delta_{kp}, \quad
\varepsilon_{ijk}\varepsilon_{ipq} = \delta_{jp}\delta_{kq} - \delta_{jq}\delta_{kp}.
\]

---

\subsection{Determinant Identity (Vector Form)}

\[
\boxed{
\det[A] = \tfrac{1}{6}\varepsilon_{ijk}\varepsilon_{pqr}A_{ip}A_{jq}A_{kr}
}
\]

Proof:

\[
\varepsilon_{ijk}\varepsilon_{pqr}A_{ip}A_{jq}A_{kr}
= \varepsilon_{ijk}\varepsilon_{ijk}\det[A]
= 6\det[A].
\]

---

\subsection{Existence of the Inverse Matrix}

A necessary and sufficient condition for the inverse of \(A\) to exist is:
\[
\det[A] \neq 0.
\]

Then:
\[
[A][A^{-1}] = [I] = [A^{-1}][A],
\quad I_{ij} = \delta_{ij}.
\]

---

\subsubsection{Inverse in Index Notation}

\[
\boxed{
A^{-1}_{ij} = \frac{1}{2\det[A]} \varepsilon_{ilm}\varepsilon_{jnp}A_{mn}A_{lp}
}
\]

Proof sketch:

Start from \(\det[A] = \varepsilon_{ijk}A_{1i}A_{2j}A_{3k}\)  
and differentiate w.r.t. \(A_{rs}\) to obtain the cofactor structure, then use:
\[
A_{ik}A^{-1}_{kj} = \delta_{ij}.
\]

---

\subsection{Summary of Key Determinant Relations}

\begingroup
\setlength{\tabcolsep}{8pt}
\renewcommand{\arraystretch}{1.15}
\begin{center}
\begin{tabular}{@{} l l @{}}
\toprule
\textbf{Identity} & \textbf{Expression} \\
\midrule
Row expansion & $\displaystyle \det[A] = \varepsilon_{ijk} A_{1i} A_{2j} A_{3k}$ \\[4pt]
Column expansion & $\displaystyle \det[A] = \varepsilon_{ijk} A_{i1} A_{j2} A_{k3}$ \\[4pt]
Transpose & $\displaystyle \det[A^T] = \det[A]$ \\[4pt]
Product & $\displaystyle \det[AB] = \det[A] \, \det[B]$ \\[4pt]
Inverse & $\displaystyle A^{-1}_{ij} = 
\frac{1}{2\,\det[A]} \, \varepsilon_{ilm} \varepsilon_{jnp} A_{mn} A_{lp}$ \\
\bottomrule
\end{tabular}
\end{center}
\endgroup

\subsection{Coordinate Transformations and Tensor Transformations}

---

\subsubsection{Coordinate Systems}

Let the base vectors of the original coordinate system be 
\[
(\mathbf{e}_1, \mathbf{e}_2, \mathbf{e}_3),
\]
and the coordinates of a point \(P\) be \(X = (X_1, X_2, X_3)\).  
Then the position vector is:
\[
\mathbf{x} = X_i \mathbf{e}_i.
\]

After a rigid rotation of the coordinate axes about the origin, the new base vectors are 
\[
(\mathbf{e}_1', \mathbf{e}_2', \mathbf{e}_3'),
\]
and the same position vector can be written as:
\[
\mathbf{x} = X_i' \mathbf{e}_i'.
\]

---

\subsubsection{Transformation Matrix}

The transformation between the two coordinate bases is given by:
\[
Q_{ij} = \mathbf{e}_i' \cdot \mathbf{e}_j = \cos(\mathbf{e}_i', \mathbf{e}_j),
\]
where \(Q_{ij}\) is the **direction cosine matrix** (orthogonal transformation matrix).

Thus:
\[
\mathbf{e}_i' = Q_{ij} \mathbf{e}_j, 
\quad
\mathbf{e}_i = Q_{ji} \mathbf{e}_j'.
\]

---

\subsubsection{Orthogonal Transformation Relations}

Since the base vectors are orthonormal:
\[
\mathbf{e}_i \cdot \mathbf{e}_j = \delta_{ij},
\quad
\mathbf{e}_i' \cdot \mathbf{e}_j' = \delta_{ij}.
\]

Using \( \mathbf{e}_i' = Q_{ij} \mathbf{e}_j \):
\[
\mathbf{e}_i' \cdot \mathbf{e}_j' = Q_{ik}Q_{jl}\mathbf{e}_k \cdot \mathbf{e}_l
= Q_{ik}Q_{jl}\delta_{kl}
= Q_{ik}Q_{jk} = \delta_{ij}.
\]

Hence:
\[
\boxed{
Q_{ik}Q_{jk} = Q_{ki}Q_{kj} = \delta_{ij}
\quad \Rightarrow \quad
[Q][Q]^T = [Q]^T[Q] = [I].
}
\]

This shows that \( [Q] \) is an orthogonal matrix and
\[
Q^{-1} = Q^T.
\]

---

\subsubsection{Transformation of Coordinates}

Since
\[
\mathbf{x} = X_i \mathbf{e}_i = X_i' \mathbf{e}_i',
\]
we have:
\[
X_i = Q_{ij}X_j',
\quad
X_i' = Q_{ji}X_j.
\]

Thus, the coordinates transform by \( [Q] \) and its transpose.

---

\subsubsection{Example: Rotation About \( \mathbf{e}_3 \)}

For a rotation by an angle \( \theta \) about \( \mathbf{e}_3 \):
\[
[Q] =
\begin{bmatrix}
\cos\theta & \sin\theta & 0 \\
-\sin\theta & \cos\theta & 0 \\
0 & 0 & 1
\end{bmatrix}.
\]

---

\subsection{Transformation of Vectors}

Let a vector \( \mathbf{a} = a_i \mathbf{e}_i = a_i' \mathbf{e}_i' \).  
Then:
\[
a_i' = Q_{ij}a_j,
\quad
a_i = Q_{ji}a_j'.
\]

Proof of magnitude invariance:
\[
|\mathbf{a}|^2 = a_i a_i = a_i' a_i' = Q_{ij}a_j Q_{ik}a_k \delta_{jk}
= a_j a_k Q_{ij}Q_{ik} = a_j a_k \delta_{jk} = a_i a_i.
\]
Hence, the magnitude is invariant under orthogonal transformations.

---

\subsubsection{Tensor (Dyadic) Product of Two Vectors}

Given vectors \( \mathbf{a} = a_i \mathbf{e}_i \) and \( \mathbf{b} = b_i \mathbf{e}_i \),
their dyadic (tensor) product is:
\[
\mathbf{a}\mathbf{b} = a_i b_j \mathbf{e}_i \mathbf{e}_j.
\]

This is a second-order tensor whose components transform as:
\[
a_i' b_j' = Q_{ik} Q_{jl} a_k b_l.
\]

Thus:
\[
\boxed{A_{ij}' = Q_{ik}Q_{jl}A_{kl}}
\]
for a general second-order tensor \(A_{ij}\).

---

\subsubsection{Cartesian Tensors of First Order}

A first-order tensor (vector) is defined by:
\[
a_i' = Q_{ij}a_j,
\quad
a_i = Q_{ji}a_j'.
\]

Matrix form:
\[
\{a'\} = [Q]\{a\},
\quad
\{a\} = [Q]^T\{a'\}.
\]

---

\subsubsection{Summary}

\begingroup
\setlength{\tabcolsep}{10pt}
\renewcommand{\arraystretch}{1.15}
\begin{center}
\begin{tabular}{@{} l l c @{}}
\toprule
\textbf{Type} & \textbf{Transformation Law} & \textbf{Tensor Order} \\
\midrule
Vector & $\displaystyle a_i' = Q_{ij} a_j$ & 1st order \\[4pt]
Second-order tensor & $\displaystyle A_{ij}' = Q_{ik} Q_{jl} A_{kl}$ & 2nd order \\[4pt]
Orthogonality & $\displaystyle Q_{ik} Q_{jk} = \delta_{ij}$ & -- \\[4pt]
Inverse relation & $\displaystyle Q^{-1} = Q^{T}$ & -- \\
\bottomrule
\end{tabular}
\end{center}
\endgroup---
\subsection{Cartesian Tensors and Tensor Algebra}

---

\subsubsection{Second-Order Cartesian Tensors (Matrix Form)}

A second-order tensor is represented by a \(3\times3\) matrix:
\[
A'_{ij} = Q_{ip} Q_{jq} A_{pq},
\quad \text{or} \quad
[A'] = [Q][A][Q]^T.
\]
Conversely:
\[
A_{ij} = Q_{pi} Q_{qj} A'_{pq}.
\]

Hence, the tensor product of two vectors produces a **second-order tensor**.

---

\subsubsection{Zero-Order Tensor (Scalar)}

A scalar is a tensor of order zero — it has the same value in all coordinate systems.  
Examples: temperature, density, specific energy.

Invariant quantity:
\[
a' = a.
\]
Something that does not change with coordinate transformation is called an **invariant**.

---

\subsubsection{Trace of a Tensor}

\[
\text{tr}(A) = A_{ii} = Q_{ip} Q_{iq} A_{pq} = \delta_{pq} A_{pq} = A_{kk}.
\]

Hence, the trace of a tensor is invariant under coordinate transformations.

---

\subsubsection{Product Between Tensors}

If
\[
C_{ij} = A_{ik} B_{kj},
\]
and
\[
b_i = A_{ij} c_j,
\]
then \(b_i\) is a vector and \(C_{ij}\) is a tensor.  
Hence the product of two tensors (or a tensor and a vector) yields another tensor.

---

\subsubsection{Identity Tensor}

\[
\mathbf{I} = \delta_{ij} \mathbf{e}_i \mathbf{e}_j =
\begin{bmatrix}
1 & 0 & 0\\
0 & 1 & 0\\
0 & 0 & 1
\end{bmatrix}.
\]

\[
\mathbf{I}\mathbf{a} = \mathbf{a}\mathbf{I} = \mathbf{a}.
\]

---

\subsubsection{Permutation Tensor of Order 3}

The permutation tensor (Levi–Civita symbol) is a third-order tensor defined as:
\[
\varepsilon'_{ijk} = \det[Q] \, Q_{ip} Q_{jq} Q_{kr} \varepsilon_{pqr}.
\]
For proper rotations (\(\det[Q]=+1\)):
\[
\varepsilon'_{ijk} = Q_{ip} Q_{jq} Q_{kr} \varepsilon_{pqr}.
\]

---

\subsubsection{Tensor Properties}

\[
\begin{aligned}
&\text{i)} \quad I \cdot a = a, \\
&\text{ii)} \quad I A = A I = A, \\
&\text{iii)} \quad A(\mathbf{b} \cdot \mathbf{c}) = (A\mathbf{b}) \cdot \mathbf{c}, \\
&\text{iv)} \quad A(\mathbf{b} \times \mathbf{c}) = (A\mathbf{b}) \times (A\mathbf{c}), \\
&\text{v)} \quad (a\mathbf{b}) \cdot c = a(\mathbf{b}\cdot c), \\
&\text{vi)} \quad (a\mathbf{b}) \cdot (\mathbf{c}\mathbf{d}) = (a\cdot \mathbf{c})(\mathbf{b}\cdot \mathbf{d}),\\
&\text{vii)} \quad (A\mathbf{a})\cdot \mathbf{b} = \mathbf{a}\cdot (A^T\mathbf{b}).
\end{aligned}
\]

---

\subsubsection{Tensor Transpose}

\[
(A^T)_{ij} = A_{ji}.
\]
\[
[A^T] = [A]^T.
\]

For tensor products:
\[
(A\mathbf{b})\cdot \mathbf{c} = \mathbf{b}\cdot (A^T \mathbf{c}).
\]

Properties:
\[
\begin{aligned}
& (A^T)^T = A, \quad (A+B)^T = A^T + B^T, \\
& (AB)^T = B^T A^T, \quad (a\mathbf{b})^T = \mathbf{b}a.
\end{aligned}
\]

---

\subsubsection{Tensor Inverse}

If \(A^{-1}\) exists:
\[
A A^{-1} = A^{-1} A = I.
\]
\[
(A^{-1})^T = (A^T)^{-1}, \quad (AB)^{-1} = B^{-1}A^{-1}.
\]

---

\subsubsection{Orthogonal Tensors}

A tensor \(R\) is **orthogonal** if:
\[
R^T R = R R^T = I.
\]
It represents a pure rotation or reflection.

Properties:
\[
|R| = 1, \quad R^{-1} = R^T, \quad \text{and} \quad |Ra| = |a|.
\]
Proof:
\[
|Ra|^2 = (Ra)\cdot (Ra) = a\cdot (R^T R)a = a\cdot a.
\]

Thus, a rotation tensor preserves vector lengths and angles.

---

\subsubsection{Tensor Symmetric and Antisymmetric Parts}

Any second-order tensor \(A\) can be decomposed as:
\[
A = \frac{1}{2}(A + A^T) + \frac{1}{2}(A - A^T),
\]
where
\[
A^{(s)} = \frac{1}{2}(A + A^T) \quad \text{(symmetric part)},
\quad
A^{(a)} = \frac{1}{2}(A - A^T) \quad \text{(antisymmetric part)}.
\]

---

\subsubsection{Tensor Invariants}

A **tensor invariant** is a quantity that remains unchanged under a coordinate transformation.  
For a second-order tensor \(A_{ij}\), the three principal invariants are:
\[
\begin{aligned}
I_1 &= \text{tr}(A) = A_{ii},\\
I_2 &= \tfrac{1}{2}\big[(\text{tr}A)^2 - \text{tr}(A^2)\big],\\
I_3 &= \det(A).
\end{aligned}
\]

---

\subsubsection{Tensor Identity and Orthogonality Example}

If \(R_{ij} = \mathbf{e}_i' \cdot \mathbf{e}_j\), then
\[
R_{ij}R_{ik} = \delta_{jk}, \quad R_{ij}R_{kj} = \delta_{ik}.
\]
Thus, rotation tensors preserve orthonormality and mutual perpendicularity of base vectors.

---
\subsubsection{Summary Table}

\begingroup
\setlength{\tabcolsep}{8pt}
\renewcommand{\arraystretch}{1.15}
\begin{center}
\begin{tabular}{@{} l l l @{}}
\toprule
\textbf{Concept} & \textbf{Definition} & \textbf{Key Relation} \\
\midrule
Tensor transformation & $\displaystyle A'_{ij} = Q_{ip} Q_{jq} A_{pq}$ & $\displaystyle [A'] = [Q][A][Q]^T$ \\[4pt]
Orthogonality & $\displaystyle Q^T Q = QQ^T = I$ & $\displaystyle Q^{-1} = Q^T$ \\[4pt]
Identity tensor & $\displaystyle I_{ij} = \delta_{ij}$ & $\displaystyle I a = a$ \\[4pt]
Symmetric / antisymmetric parts & $\displaystyle A = A^{(s)} + A^{(a)}$ & $\displaystyle A^{(s)} = \tfrac{1}{2}(A + A^T)$ \\[4pt]
Invariants & $\displaystyle I_1,\, I_2,\, I_3$ & invariant under rotation \\
\bottomrule
\end{tabular}
\end{center}
\endgroup
---
\subsection{Tensor Invariants, Eigenvalues, and Differential Operators}

---

\subsubsection{First-Order Tensor (Vector)}

Magnitude:
\[
a_i a_i = Q_{ij} a_j Q_{ik} a_k = \delta_{jk} a_j a_k = a_j a_j.
\]
Hence, the magnitude is invariant under rotation.

---

\subsubsection{Second-Order Tensor (Matrix)}

Trace:
\[
\text{tr}(A) = A_{ii}, \quad \text{tr}(A^2) = A_{ij} A_{ji}.
\]

Proof:
\[
(A^2)_{ij} = A_{ik} A_{kj},
\quad
(A^3)_{ij} = A_{ik} A_{kl} A_{lj}.
\]

Hence:
\[
\text{tr}(A^n) = A_{i_1 i_2} A_{i_2 i_3} \dots A_{i_n i_1}.
\]

---

\subsubsection{Third-Order Tensor Contraction}

\[
T_{ijk} \delta_{jk} = T_{i jj}.
\]

---

\subsubsection{Tensor Invariants}

For any second-order tensor \(A_{ij}\):
\[
\begin{aligned}
I_1 &= \text{tr}(A) = A_{ii}, \\
I_2 &= \tfrac{1}{2}[(\text{tr}A)^2 - \text{tr}(A^2)] = \tfrac{1}{2}(A_{ii}A_{jj} - A_{ij}A_{ji}),\\
I_3 &= \det(A).
\end{aligned}
\]

Expanded determinant in index notation:
\[
\det[A] = \tfrac{1}{6}\varepsilon_{ijk}\varepsilon_{pqr}A_{ip}A_{jq}A_{kr}.
\]

---

\subsubsection{Properties of Determinants}

\[
\begin{aligned}
&\text{i)} && \det[I] = 1,\\
&\text{ii)} && \det[A^T] = \det[A],\\
&\text{iii)} && \det(aA) = a^3 \det[A],\\
&\text{iv)} && \det[AB] = \det[A]\det[B],\\
&\text{v)} && \det[A^{-1}] = \dfrac{1}{\det[A]},\\
&\text{vi)} && \det[Q] = \pm 1 \quad \text{if $Q$ is orthogonal.}
\end{aligned}
\]

---

\subsubsection{Adjugate (Adjoint) Tensor}

\[
A^*_{ij} = \tfrac{1}{2}\varepsilon_{ipq}\varepsilon_{jmn}A_{pm}A_{qn}.
\]

It satisfies:
\[
A A^* = A^* A = (\det A) I.
\]

Hence:
\[
A^{-1} = \dfrac{A^*}{\det A}.
\]

---

\subsubsection{Eigenvalues and Eigenvectors}

\[
A n = \lambda n,
\quad n \cdot n = 1.
\]
Here, \(\lambda\) is the eigenvalue and \(n\) the corresponding eigenvector.

The characteristic equation:
\[
\det(A - \lambda I) = 0.
\]
Expanding gives:
\[
-\lambda^3 + I_1 \lambda^2 - I_2 \lambda + I_3 = 0.
\]

At least one eigenvalue is real.  
Eigenvalues are invariant under rotation (coordinate-independent).

---

\subsubsection{Cayley–Hamilton Theorem}

Every second-order tensor satisfies its own characteristic equation:
\[
A^3 - I_1 A^2 + I_2 A - I_3 I = 0.
\]

Example:
\[
A = 
\begin{bmatrix}
5 & 4 & 0\\
4 & -1 & 0\\
0 & 0 & 3
\end{bmatrix}.
\]
\[
\det(A - \lambda I) = -(\lambda - 7)(\lambda - 3)(\lambda + 3) = 0.
\]
Thus, \(\lambda_1 = 7,\ \lambda_2 = 3,\ \lambda_3 = -3.\)

---

\subsubsection{Symmetric Tensors}

For symmetric tensors \(A = A^T\):
\[
A n = \lambda n.
\]

Real symmetric tensors have three real (not necessarily distinct) eigenvalues.  
Eigenvectors associated with distinct eigenvalues are orthogonal.

---

\subsubsection{Orthogonality of Eigenvectors}

Let
\[
A n^{(1)} = \lambda_1 n^{(1)}, \quad A n^{(2)} = \lambda_2 n^{(2)}.
\]
Then:
\[
n^{(1)} \cdot A n^{(2)} = \lambda_2 n^{(1)} \cdot n^{(2)}, \quad
A n^{(1)} \cdot n^{(2)} = \lambda_1 n^{(1)} \cdot n^{(2)}.
\]
Subtracting gives:
\[
(\lambda_1 - \lambda_2) n^{(1)} \cdot n^{(2)} = 0.
\]
Thus, \(n^{(1)} \perp n^{(2)}\).

---

\subsubsection{Root Multiplicity}

If \(\lambda_1, \lambda_2, \lambda_3\) are distinct, there exist three orthogonal eigenvectors.

If \(\lambda_1 = \lambda_2 \neq \lambda_3\), any vector in the plane normal to \(n^{(3)}\) is an eigenvector with eigenvalue \(\lambda_1\).

If \(\lambda_1 = \lambda_2 = \lambda_3 = \lambda\), then:
\[
A = \lambda I,
\]
which defines an **isotropic tensor**.

---

\subsubsection{Principal Axes Theorem}

A real symmetric tensor can be diagonalized by a rotation:
\[
[A] = 
\begin{bmatrix}
\lambda_1 & 0 & 0\\
0 & \lambda_2 & 0\\
0 & 0 & \lambda_3
\end{bmatrix}.
\]
Hence, symmetric tensors can always be rotated into a coordinate system where they are diagonal.

Spectral representation:
\[
A = \lambda_1 n^{(1)} n^{(1)} + \lambda_2 n^{(2)} n^{(2)} + \lambda_3 n^{(3)} n^{(3)}.
\]

---

\subsubsection{Gradient of a Scalar Field}

For a scalar function \(\phi(x_1, x_2, x_3)\):
\[
d\phi = \frac{\partial \phi}{\partial x_i} dx_i = \nabla \phi \cdot d\mathbf{x}.
\]
Thus:
\[
\nabla \phi = \frac{\partial \phi}{\partial x_i} \mathbf{e}_i.
\]

---

\subsubsection{Directional Derivative}

If \(\mathbf{e}\) is a unit vector:
\[
\frac{d\phi}{ds} = \nabla \phi \cdot \mathbf{e}.
\]
The gradient of a scalar gives the direction of **maximum rate of increase** of \(\phi\).

---

\subsubsection{Gradient of a Vector Field}

For a vector field \(\mathbf{a} = a_i \mathbf{e}_i\):
\[
\nabla \mathbf{a} = \frac{\partial a_i}{\partial x_j} \mathbf{e}_i \mathbf{e}_j.
\]

---

\subsection{Summary Table of Tensor Properties}

\begingroup
\setlength{\tabcolsep}{8pt}
\renewcommand{\arraystretch}{1.15}
\begin{center}
\begin{tabular}{@{} l l l @{}}
\toprule
\textbf{Concept} & \textbf{Definition} & \textbf{Example Expression} \\
\midrule
First invariant & $\displaystyle I_1 = \text{tr}(A)$ & $\displaystyle A_{ii}$ \\[4pt]
Second invariant & $\displaystyle I_2 = \tfrac{1}{2}\big[(\text{tr}A)^2 - \text{tr}(A^2)\big]$ & \\[4pt]
Third invariant & $\displaystyle I_3 = \det(A)$ & $\displaystyle \tfrac{1}{6}\varepsilon_{ijk}\varepsilon_{pqr}A_{ip}A_{jq}A_{kr}$ \\[4pt]
Spectral decomposition & $\displaystyle A = \sum_i \lambda_i n^{(i)} n^{(i)}$ & principal directions \\[4pt]
Isotropic tensor & $\displaystyle A = \lambda I$ & identical eigenvalues \\[4pt]
Gradient & $\displaystyle \nabla \phi = \partial_i \phi\, e_i$ & rate of change of scalar \\
\bottomrule
\end{tabular}
\end{center}
\endgroup

\section{Motion of a Continuum}

Suppose a body $B$ is composed of a set of particles, and at each instant of time each particle of the set is assigned a unique point of a closed region $R$ of the 3D Euclidean space, such that each point of $R$ is occupied by just one particle.  
We call $R$ the \textit{configuration of body $B$ at time $t$}.

\vspace{8pt}

We label the particles by a position vector $\underline{X} = (X_1, X_2, X_3)$ referred to the fixed Cartesian axes.

% \begin{center}
% \begin{tikzpicture}[scale=1.2]
%   % reference config
%   \draw[thick] (-3,0) ellipse (1.2 and 0.7);
%   \node at (-3.3,0.9) {$B_0$};
%   \node at (-3.8,-0.6) {$t=0$};
%   \draw[->, thick] (-3,0) -- (-1.8,0.3) node[midway, above] {$\underline{X}$};
%   \filldraw (-1.8,0.3) circle (1.5pt) node[above right] {$P_0$};

%   % current config
%   \draw[thick] (2,0.1) ellipse (1.2 and 0.7);
%   \node at (2.3,0.9) {$B_t$};
%   \node at (2.9,-0.6) {$t>0$};
%   \draw[->, thick] (2,0.1) -- (3.1,0.4) node[midway, above] {$\underline{x}$};
%   \filldraw (3.1,0.4) circle (1.5pt) node[above right] {$P$};

%   % mapping arrow
%   \draw[->, thick, blue!60!black] (-1.8,0.3) .. controls (0.2,1.0) .. (3.1,0.4);
%   \node[above, blue!60!black] at (0.6,1.1) {$\underline{x} = \underline{\chi}(\underline{X}, t)$};
%   \draw[->] (-4.5,-1) -- (4,-1) node[right] {$O$};
% \end{tikzpicture}
% \end{center}

The material which occupies the region $B_0$ at $t=0$ moves and occupies a new region $B_t$ at time $t$.  
The motion of the body can be described by specifying the dependence of the position $\underline{x}$ of all particles of the body at time $t$ on the position $\underline{X}$ at time $t=0$:
\begin{align}
\underline{x} = \underline{\chi}(\underline{X}, t)
\end{align}

Coordinates $\underline{X}$ label the particles $\Rightarrow$ \textbf{material coordinates}.  
Coordinates $\underline{x}$ identify points in space which, in general, are occupied by different particles at different times $\Rightarrow$ \textbf{spatial coordinates}.

\vspace{8pt}
\noindent
Configuration of body:
\begin{itemize}
  \item at $t = 0$: \textit{reference configuration}
  \item at $t = t$: \textit{current configuration}
\end{itemize}

\subsection{Material and Spatial Descriptions}

\textbf{Material description (Lagrangian description):}
\begin{align}
\rho(\underline{X}, t) &: \text{density} \\[6pt]
\underline{v}(\underline{X}, t) &: \text{velocity of particles at time } t, 
\text{ defined by their material coordinate } \underline{X} \text{ at } t=0
\end{align}

\textbf{Spatial description (Eulerian description):}
\begin{align}
\rho(\underline{x}, t) &: \text{density} \\[6pt]
\underline{v}(\underline{x}, t) &: \text{velocity of the particle which, at time } t, 
\text{ occupies the position } \underline{x}
\end{align}

\vspace{8pt}
The mapping $\underline{x} = \underline{\chi}(\underline{X}, t)$ 
has a unique inverse $\underline{X} = \underline{\chi}^{-1}(\underline{x}, t)$,
since only one particle can occupy one place at one time.

\vspace{10pt}
\textit{Remark:}  
The spatial (Eulerian) description does not provide direct information 
regarding changes in particle properties as they move about.

\subsection{Displacement and Velocity}

\textbf{Displacement:}
\begin{align}
\underline{u} = \underline{x} - \underline{X}
\end{align}

In the \textbf{material (Lagrangian) description}:
\begin{align}
\underline{u}(\underline{X}, t) = \underline{\chi}(\underline{X}, t) - \underline{X}
\end{align}

In the \textbf{spatial (Eulerian) description}:
\begin{align}
\underline{u}(\underline{x}, t) = \underline{x} - \underline{\chi}^{-1}(\underline{x}, t)
\end{align}

\vspace{6pt}
The \textbf{velocity} $\underline{v}$ of a particle is the rate of change of its displacement:
\begin{align}
\underline{v}(\underline{X}, t) 
= \frac{\partial \underline{u}(\underline{X}, t)}{\partial t}
= \frac{\partial \underline{\chi}(\underline{X}, t)}{\partial t}
\bigg|_{\underline{X}}
\end{align}

\subsection{Material Derivative}

The time rate of change of a quantity (such as temperature, density, or velocity) 
of a material particle is known as the \textbf{material derivative}.

\vspace{6pt}
\textbf{In material coordinates (Lagrangian):}
\begin{align}
\phi = \phi(\underline{X}, t)
\quad \Rightarrow \quad
\frac{d\phi}{dt} = \dot{\phi}
\end{align}

\textbf{In spatial coordinates (Eulerian):}
\begin{align}
\phi = \phi(\underline{x}, t)
\end{align}

Applying the chain rule:
\begin{align}
\frac{d\phi}{dt} 
&= \frac{\partial \phi}{\partial t}
 + \frac{\partial \phi}{\partial x_1}\frac{dx_1}{dt}
 + \frac{\partial \phi}{\partial x_2}\frac{dx_2}{dt}
 + \frac{\partial \phi}{\partial x_3}\frac{dx_3}{dt} \\[6pt]
&= \frac{\partial \phi}{\partial t}
 + \frac{\partial \phi}{\partial x_i}\frac{dx_i}{dt}
\qquad \text{(Einstein summation)} \\[6pt]
&= \frac{\partial \phi}{\partial t} 
 + (\nabla \phi) \cdot \underline{v}
\end{align}

\textbf{Therefore:}
\begin{align}
\frac{d\phi}{dt} 
= \left( \frac{\partial \phi}{\partial t} \right)_{\underline{x}}
+ \nabla \phi \cdot \underline{v}
\end{align}

For any vector field $\underline{u} = \underline{u}(\underline{x}, t)$:
\begin{align}
\frac{d\underline{u}}{dt}
&= \frac{\partial \underline{u}}{\partial t}
 + (\underline{v} \cdot \nabla)\underline{u} \\[4pt]
\Rightarrow 
\left(\frac{du_i}{dt}\right)
&= \frac{\partial u_i}{\partial t}
 + v_j \frac{\partial u_i}{\partial x_j}
\end{align}

\begin{example} 
A body undergoes a motion defined by
\begin{align}
x_1 &= \frac{X_1}{1 + a^2 t^2}, \qquad
x_2 = X_2, \qquad
x_3 = X_3
\end{align}
where $a$ is a constant.

Find the displacement and velocity fields.

\vspace{4pt}
\textbf{In material description:}
\begin{align}
u_1 &= x_1 - X_1 = X_1\!\left(\frac{1}{1 + a^2 t^2} - 1\right)
= -\frac{a^2 t^2 X_1}{1 + a^2 t^2} \\[4pt]
u_2 &= 0, \qquad u_3 = 0
\end{align}

\textbf{Velocity (material form):}
\begin{align}
v_1 &= \frac{\partial u_1}{\partial t}\bigg|_{\underline{X}}
= \frac{-2a^2 t X_1}{(1 + a^2 t^2)^2} \\[4pt]
v_2 &= 0, \qquad v_3 = 0
\end{align}

\vspace{6pt}
\textbf{In spatial description:}
Using $X_1 = x_1(1 + a^2 t^2)$,
\begin{align}
u_1(x_1, t) &= -a^2 t^2 x_1, \qquad 
u_2 = 0, \quad u_3 = 0
\end{align}

\textbf{Velocity (Eulerian form):}
\begin{align}
v_1 
&= \frac{\partial u_1}{\partial t}\bigg|_{\underline{x}}
+ \frac{\partial u_1}{\partial x_j}\bigg|_t v_j \\[4pt]
&= \frac{\partial (-a^2 t^2 x_1)}{\partial t} 
+ \frac{\partial (-a^2 t^2 x_1)}{\partial x_1} v_1 \\[6pt]
&= -2a^2 t x_1 - a^2 t^2 v_1 \\[4pt]
\Rightarrow \quad v_1 
&= \frac{-2a^2 t x_1}{1 + a^2 t^2}, 
\qquad v_2 = v_3 = 0
\end{align}
\end{example}
\subsection{Deformation Analysis}

\subsubsection{(i) At any time $t$}

An infinitesimal element $d\underline{X}$ in the reference configuration becomes
$d\underline{x}$ in the current configuration at time $t$.

\begin{align}
\underline{x} = \underline{\chi}(\underline{X}, t)
\end{align}

Taking the differential:
\begin{align}
d\underline{x} 
= \frac{\partial \underline{x}}{\partial \underline{X}} \, d\underline{X}
\end{align}

We write
\begin{align}
d\underline{x} = \underline{F} \, d\underline{X}
\end{align}
where
\begin{align}
F_{ij} = \frac{\partial x_i}{\partial X_j},
\qquad 
\underline{F} = \frac{\partial \underline{x}}{\partial \underline{X}}
\end{align}

\noindent
$\underline{F}$ is called the \textbf{deformation gradient}.
\begin{align}
\underline{F} =
\begin{bmatrix}
\frac{\partial x_1}{\partial X_1} & \frac{\partial x_1}{\partial X_2} & \frac{\partial x_1}{\partial X_3} \\[4pt]
\frac{\partial x_2}{\partial X_1} & \frac{\partial x_2}{\partial X_2} & \frac{\partial x_2}{\partial X_3} \\[4pt]
\frac{\partial x_3}{\partial X_1} & \frac{\partial x_3}{\partial X_2} & \frac{\partial x_3}{\partial X_3}
\end{bmatrix}
\end{align}

To define strains, we compare $d\underline{x}$ with $d\underline{X}$.
Although $\underline{F}$ is a fundamental tensor in deformation analysis,
it is not a suitable measure of deformation since it also contains the rigid-body motion.

---

\subsubsection{(ii) Following a material particle ($\underline{X} = \text{const}$)}

At a point $\underline{X}$:
\begin{align}
d\underline{x} 
= \frac{\partial \underline{x}(\underline{X}, t)}{\partial t}\,dt 
= \underline{v}\,dt
\end{align}

We can also write
\begin{align}
d\underline{X} = \underline{F}^{-1} \, d\underline{x}
\end{align}

Define $J$ as the determinant of $\underline{F}$:
\begin{align}
J = \det[\underline{F}]
\end{align}

\noindent
$J$ is the \textbf{Jacobian} of the deformation.

\vspace{4pt}
Notice $J \neq 0$ ensures a local inverse exists and $J = 1$ at $t = 0$.
During motion $J > 0$.

By mass conservation,
\begin{align}
\rho J = \rho_0
\end{align}
where $\rho_0$ is the reference density and $\rho$ is the current density.

---

\subsubsection{Change in length}

At any point,
\begin{align}
ds_0 = \| d\underline{X} \|,
\qquad
ds = \| d\underline{x} \|
\end{align}

Let $\lambda = ds/ds_0$ be the \textbf{stretch ratio}.
Then,
\begin{align}
ds^2 = d\underline{x} \cdot d\underline{x}
= ( \underline{F} \, d\underline{X} ) \cdot ( \underline{F} \, d\underline{X} )
= d\underline{X} \cdot (\underline{F}^T \underline{F}) \, d\underline{X}
\end{align}

Define
\begin{align}
\underline{C} = \underline{F}^T \underline{F}
\end{align}
\noindent
$\underline{C}$ is the \textbf{Right Cauchy--Green deformation tensor.}

---

\noindent
\textbf{Properties of $\underline{C}$:}
\begin{itemize}
  \item $\underline{C}$ is symmetric: $\underline{C} = \underline{C}^T$.
  \item $\underline{C}$ is positive definite because $ds^2 > 0$
  for all nonzero $d\underline{X}$.
  \item It maps a finite-length vector into another finite-length vector.
\end{itemize}

\noindent
If $\underline{a}$ is an arbitrary vector, then
\begin{align}
\underline{a} \cdot (\underline{C}\underline{a}) 
= (\underline{F}\underline{a}) \cdot (\underline{F}\underline{a}) > 0
\end{align}
so $\underline{C}$ is \textbf{positive definite.}

A tensor $\underline{A}$ is positive definite if 
$\underline{a} \cdot (\underline{A}\underline{a}) > 0$ 
for all nonzero $\underline{a}$.

---

\subsubsection{Configuration comparison and mapping}

\begin{itemize}
  \item $\underline{x} = \underline{\chi}(\underline{X}, t)$ represents the motion.
  \item For any fixed $t$, $\underline{x} = \underline{\chi}(\underline{X})$ 
  defines a deformation from the reference configuration.
  \item $\underline{F} = \partial \underline{x} / \partial \underline{X}$ is nonsingular 
  if $\det(\underline{F}) \neq 0$, ensuring the mapping is one-to-one.
\end{itemize}

We have
\begin{align}
d\underline{x} = \underline{F} \, d\underline{X}, 
\qquad
\underline{F}^T \underline{F} = \underline{C}
\end{align}

\noindent
Because $\underline{F}^T \underline{F}$ is symmetric and positive definite, 
$\underline{C}$ serves as a proper measure of deformation independent of rigid-body motion.

\[
\underline{F}^T \underline{F} = \text{positive definite}, 
\qquad
\underline{C} = \underline{F}^T \underline{F}
\]

\subsection{Rotation Tensor}

Every rotation tensor $\underline{R}$ has an eigenvalue of $1$.

\begin{proof}
\begin{align}
\underline{R} \underline{R}^T &= \underline{I}, 
\qquad \det(\underline{R}) = +1
\end{align}

Then,
\begin{align}
\underline{R}^T \underline{R} &= \underline{I} \\[6pt]
(\underline{R}^T - \underline{I})(\underline{R} - \underline{I})
&= \underline{R}^T \underline{R} - \underline{R}^T - \underline{R} + \underline{I} \\[6pt]
&= \underline{I} - \underline{R}^T - \underline{R} + \underline{I} \\[6pt]
&= 2\underline{I} - (\underline{R} + \underline{R}^T)
\end{align}

Taking determinants on both sides (for a $3 \times 3$ matrix):
\begin{align}
\det[(\underline{R} - \underline{I})(\underline{R}^T - \underline{I})]
&= \det(\underline{R} - \underline{I}) \det(\underline{R}^T - \underline{I}) \\[4pt]
&= \det(\underline{R} - \underline{I}) \det(\underline{R} - \underline{I})^T \\[4pt]
&= [\det(\underline{R} - \underline{I})]^2
\end{align}

Since $\det(\underline{R}) = +1$,
\begin{align}
\det(\underline{I} - \underline{R}) 
&= -\det(\underline{R} - \underline{I})
\end{align}

Thus,
\begin{align}
\det(\underline{R} - \underline{I}) = 0
\end{align}

Hence, $\lambda = 1$ is one of the eigenvalues of $\underline{R}$.

If the corresponding eigenvector associated with $\lambda = 1$ is $\underline{n}$, then
\begin{align}
\underline{R}\,\underline{n} = \underline{n}
\end{align}
which represents the \textbf{axis of rotation}.
\end{proof}

\subsubsection{Rotation Tensor — Orthogonal Matrix Form}

Let $\underline{R}$ be an orthogonal matrix such that
\begin{align}
\underline{R}\underline{R}^T = \underline{I}
\end{align}

Example in two dimensions:
\begin{align}
\underline{R} =
\begin{bmatrix}
\cos\theta & -\sin\theta \\[4pt]
\sin\theta & \cos\theta
\end{bmatrix}
\end{align}

\begin{proof}
\begin{align}
\underline{R}\underline{R}^T &= \underline{I} \\[6pt]
\Rightarrow \;
(\underline{R} - \underline{I})(\underline{R}^T - \underline{I})
&= -(\underline{R} - \underline{I})^T \\[6pt]
\det[(\underline{R} - \underline{I})(\underline{R}^T - \underline{I})]
&= (-1)^n \det[\underline{R} - \underline{I}]
\end{align}

Hence,
\begin{align}
\det[\underline{R} - \underline{I}] \, \det[\underline{R}^T - \underline{I}]
= (-1)^n \det[\underline{R} - \underline{I}]
\end{align}

For a matrix of dimension $n$:
\begin{itemize}
  \item If $n$ is \textbf{odd}: $(-1)^n = -1$,  
        then $\det[\underline{R} - \underline{I}] = 0$.
  \item If $n$ is \textbf{even}: $(-1)^n = +1$,  
        then $\det[\underline{R} - \underline{I}] = \det[\underline{R}^T - \underline{I}]$,  
        and we \textit{cannot conclude} that $\det[\underline{R} - \underline{I}] = 0$.
\end{itemize}

Therefore, in odd-dimensional space (e.g.\ $n=3$),  
every rotation tensor $\underline{R}$ necessarily has an eigenvalue of $1$.
\end{proof}

\subsubsection{Rotation Tensor: Axis of Rotation and Matrix Form}

Let $\underline{n}$ be the unit eigenvector of a rotation tensor $\underline{R}$ 
associated with the eigenvalue $\lambda = 1$, that is:
\begin{align}
\underline{R}\underline{n} = \underline{n}
\end{align}

---

\textbf{Constructing the orthonormal basis:}

Consider an arbitrary orthonormal basis 
$\{\underline{e}_1, \underline{e}_2, \underline{e}_3\}$ with $\underline{e}_3 = \underline{n}$.
Then there exists an angle $\theta$ such that
\begin{align}
\underline{R}\underline{e}_1 &= \cos\theta\,\underline{e}_1 
+ \sin\theta\,\underline{e}_2 \\[4pt]
\underline{R}\underline{e}_2 &= -\sin\theta\,\underline{e}_1 
+ \cos\theta\,\underline{e}_2 \\[4pt]
\underline{R}\underline{e}_3 &= \underline{e}_3
\end{align}

Hence, the rotated basis vectors are:
\begin{align}
\underline{e}_1' = \cos\theta\,\underline{e}_1 + \sin\theta\,\underline{e}_2,
\qquad
\underline{e}_2' = -\sin\theta\,\underline{e}_1 + \cos\theta\,\underline{e}_2,
\qquad
\underline{e}_3' = \underline{e}_3
\end{align}

---

\textbf{Derivation:}

Given $\underline{n}$, we choose $\underline{e}_1 \perp \underline{n}$ 
and define $\underline{e}_2 = \underline{n} \times \underline{e}_1$.
This defines a second orthonormal basis $\{\underline{e}_1, \underline{e}_2, \underline{n}\}$.

Because $\underline{R}$ leaves $\underline{n}$ invariant, it rotates 
$\underline{e}_1$ and $\underline{e}_2$ in the plane orthogonal to $\underline{n}$ 
by an angle $\theta$.  
Thus, the rotation matrix relative to this basis is:
\begin{align}
[\underline{R}] =
\begin{bmatrix}
\cos\theta & -\sin\theta & 0 \\[4pt]
\sin\theta & \cos\theta  & 0 \\[4pt]
0          & 0           & 1
\end{bmatrix}
\end{align}

---

\textbf{Component relations:}
\begin{align}
R_{11} &= \cos\theta, 
& R_{12} &= -\sin\theta, 
& R_{13} &= 0, \\[4pt]
R_{21} &= \sin\theta, 
& R_{22} &= \cos\theta, 
& R_{23} &= 0, \\[4pt]
R_{31} &= 0, 
& R_{32} &= 0, 
& R_{33} &= 1
\end{align}

---

\textbf{Tensor form:}
\begin{align}
\underline{R}
&= \cos\theta\,(\underline{e}_1 \underline{e}_1 
+ \underline{e}_2 \underline{e}_2)
+ \sin\theta\,(\underline{e}_2 \underline{e}_1 
- \underline{e}_1 \underline{e}_2)
+ \underline{e}_3 \underline{e}_3
\end{align}

\textbf{Trace relation:}
\begin{align}
\operatorname{tr}[\underline{R}] = 2\cos\theta + 1
\end{align}

Thus, the rotation tensor about $\underline{n}$ 
rotates vectors in the plane orthogonal to $\underline{n}$ by angle $\theta$ 
and leaves $\underline{n}$ invariant.

\subsubsection{Geometric Interpretation of the Rotation Tensor}

The unit eigenvector $\underline{n}$ of the rotation tensor $\underline{R}$ 
defines the \textbf{axis of rotation}, and the angle $\theta$ defines the 
\textbf{angle of rotation} of the tensor $\underline{R}$.

In other words, when $\underline{R}$ operates on a vector $\underline{a}$, 
the resulting vector $\underline{R}\underline{a}$ is obtained by rotating 
$\underline{a}$ about $\underline{n}$ through an angle $\theta$.

---

\begin{proof}
Consider the same orthonormal basis 
$\{\underline{e}_1, \underline{e}_2, \underline{e}_3\}$ 
as before, with $\underline{e}_3 = \underline{n}$.

Let $\underline{a}$ have components 
$a_i$ with respect to $\underline{e}_i$:
\begin{align}
\underline{a} = a_i \underline{e}_i
\end{align}

Then,
\begin{align}
\underline{R}\underline{a} 
= a_i\,\underline{R}\underline{e}_i 
= a_i\,\underline{e}_i'
\end{align}
where $\underline{e}_i'$ are the rotated basis vectors given by
\begin{align}
\underline{e}_1' &= \cos\theta\,\underline{e}_1 + \sin\theta\,\underline{e}_2, \\[4pt]
\underline{e}_2' &= -\sin\theta\,\underline{e}_1 + \cos\theta\,\underline{e}_2, \\[4pt]
\underline{e}_3' &= \underline{e}_3
\end{align}

Thus, the components of $\underline{R}\underline{a}$ 
with respect to $\{\underline{e}_i'\}$ are the same as the components of $\underline{a}$ 
with respect to $\{\underline{e}_i\}$.

Since $\underline{e}_1$ and $\underline{e}_2$ result from rotating 
$\underline{e}_1'$ and $\underline{e}_2'$ by $\theta$ about $\underline{n}$, 
we conclude that the action of $\underline{R}$ on $\underline{a}$ 
rotates $\underline{a}$ by the angle $\theta$ about $\underline{n}$.
\end{proof}

\subsubsection{Rigid Body Motion}

\textbf{Rigid Body Translation:}
\begin{align}
\underline{x} = \underline{X} + \underline{c}(t),
\qquad \text{where } \underline{c}(0) = \underline{0}
\end{align}

\textbf{Rigid Body Rotation:}
\begin{align}
\underline{x} = \underline{R}\underline{X}
\end{align}

\textbf{General Rigid Body Motion (RBM):}
\begin{align}
\underline{x} = \underline{R}\underline{X} + \underline{c}(t)
\end{align}

This represents a combination of rotation and translation.

---

\textbf{Property:}  
For a rigid body motion, the distances between any two material points remain constant:
\begin{align}
\underline{x}_1 &= \underline{R}\underline{X}_1 + \underline{c} \\[4pt]
\underline{x}_2 &= \underline{R}\underline{X}_2 + \underline{c} \\[4pt]
\Rightarrow \quad 
\|\underline{x}_1 - \underline{x}_2\|
&= \|\underline{R}(\underline{X}_1 - \underline{X}_2)\|
= \|\underline{X}_1 - \underline{X}_2\|
\end{align}

Hence, rigid body motion preserves lengths and angles.

---

\begin{proof}
For $\underline{x} = \underline{R}\underline{X} + \underline{c}$,
\begin{align}
F_{ij} 
&= \frac{\partial x_i}{\partial X_j}
= \frac{\partial (R_{ik} X_k + c_i)}{\partial X_j}
= R_{ik} \frac{\partial X_k}{\partial X_j} \\[4pt]
&= R_{ik} \delta_{kj}
= R_{ij}
\end{align}

Thus, the deformation gradient for a rigid body motion is:
\begin{align}
\underline{F} = \underline{R}
\end{align}

Since $\underline{R}$ is an orthogonal tensor,
\begin{align}
\underline{F}^T \underline{F} = \underline{I}
\end{align}

which confirms that rigid body motion produces no strain or deformation.
\end{proof}

\subsection{Principal Stretches and Directions}

Since we require $\det(\underline{F}) > 0$, we conclude that $\det(\underline{C}) > 0$, 
which is indeed the case because $\underline{F}$ is proper orthogonal for a rigid body motion.

Also, since $\underline{F} = \underline{R}$,
\begin{align}
\underline{C} = \underline{F}^T \underline{F}
= \underline{R}^T \underline{R}
= \underline{I}
\end{align}

Thus, for a rigid motion, $\underline{C}$ is constant and equals $\underline{I}$, 
confirming that $\underline{C}$ is a suitable measure of deformation.

---

\textbf{Stretch Ratio:}

Consider a material line element $d\underline{X}$ in the reference configuration 
and $d\underline{x}$ in the current configuration.

\[
d\underline{x} = \underline{F} d\underline{X}
\]

Let 
\[
ds_0 = \|d\underline{X}\|, \qquad ds = \|d\underline{x}\|
\]
and define the stretch ratio as
\[
\lambda = \frac{ds}{ds_0}
\]

Since $ds^2 = d\underline{x}\cdot d\underline{x}$,
\begin{align}
ds^2 &= d\underline{X}\cdot \underline{C}\, d\underline{X}
\end{align}

Let $\underline{N}$ be the unit vector along $d\underline{X}$:
\[
d\underline{X} = ds_0\,\underline{N}
\]
Then
\begin{align}
\lambda^2 = \underline{N}\cdot \underline{C}\, \underline{N}
\end{align}

---

\textbf{Principal Stretches and Directions:}

To find stationary values of $\lambda^2$, 
maximize or minimize $\underline{N}\cdot\underline{C}\,\underline{N}$ 
under the constraint $\underline{N}\cdot\underline{N} = 1$.

Using Lagrange multipliers:
\begin{align}
\underline{C}\underline{N} - \lambda^2 \underline{N} = \underline{0}
\end{align}

Hence, the eigenvalues of $\underline{C}$ are $\lambda_1^2, \lambda_2^2, \lambda_3^2$,
and the corresponding eigenvectors $\underline{N}^{(1)}, \underline{N}^{(2)}, \underline{N}^{(3)}$
are the \textbf{principal directions of stretch}.

The stationary values of $\lambda^2$ are the eigenvalues of $\underline{C}$, and
\[
\lambda_i = \sqrt{\text{eigenvalue of }\underline{C}}
\]

The directions $\underline{N}^{(i)}$ define the \textbf{principal directions}.

The minimum of $\underline{N}\cdot\underline{C}\,\underline{N}$ 
equals the smallest eigenvalue of $\underline{C}$, 
and the maximum equals the largest eigenvalue.

---

\textbf{Orthogonality:}

If $\underline{C}$ is symmetric, then its eigenvectors are orthogonal:
\begin{align}
\underline{N}^{(i)} \cdot \underline{N}^{(j)} = 0, 
\quad i \ne j
\end{align}

Proof:
\begin{align}
\underline{C}\underline{N}^{(i)} &= \lambda_i^2 \underline{N}^{(i)} \\[4pt]
\underline{C}\underline{N}^{(j)} &= \lambda_j^2 \underline{N}^{(j)}
\end{align}
Taking the dot product of the first with $\underline{N}^{(j)}$ 
and the second with $\underline{N}^{(i)}$, we obtain
\begin{align}
(\lambda_i^2 - \lambda_j^2)(\underline{N}^{(i)} \cdot \underline{N}^{(j)}) = 0
\end{align}
so $\underline{N}^{(i)}$ and $\underline{N}^{(j)}$ are orthogonal if $\lambda_i \ne \lambda_j$.

---

\textbf{Relation between stretches and $\underline{C}$:}
\begin{align}
\lambda_i^2 = \underline{N}^{(i)} \cdot \underline{C}\, \underline{N}^{(i)}
\end{align}

If $\underline{N} = \underline{e}_i$ (aligned with a principal direction):
\[
\lambda_i = \sqrt{C_{ii}}
\]

If $\underline{N}^{(i)}$ and $\underline{N}^{(j)}$ are distinct,
the angle between them satisfies
\begin{align}
\cos\alpha_{ij} 
= \frac{\underline{N}^{(i)}\cdot \underline{C}\,\underline{N}^{(j)}}
{\lambda_i \lambda_j}
= \frac{C_{ij}}{\sqrt{C_{ii} C_{jj}}}
\end{align}

\subsubsection{Square Root Theorem}

Let $\underline{C}$ be a symmetric and positive definite tensor.  
Then there exists a unique symmetric, positive definite tensor $\underline{U}$ such that
\begin{align}
\underline{U}^2 = \underline{C}
\end{align}
We write this as
\[
\underline{C} = \underline{U}^2
\quad \text{or} \quad
\underline{U} = \sqrt{\underline{C}}
\]

---

\textbf{Spectral Representation:}

If 
\[
\underline{C} = \sum_{i=1}^{3} \lambda_i \, \underline{N}^{(i)} \otimes \underline{N}^{(i)}
\]
where $\lambda_i$ and $\underline{N}^{(i)}$ are the eigenvalues and eigenvectors of $\underline{C}$,  
then
\[
\underline{U} = \sum_{i=1}^{3} \sqrt{\lambda_i} \, \underline{N}^{(i)} \otimes \underline{N}^{(i)}
\]

Thus, $\underline{U}$ shares the same principal directions as $\underline{C}$, 
and its eigenvalues are $\sqrt{\lambda_i}$.

---

\textbf{Matrix Example:}

\[
\underline{C} =
\begin{bmatrix}
\lambda_1 & 0 \\[4pt]
0 & \lambda_2
\end{bmatrix},
\qquad
\underline{U} =
\begin{bmatrix}
\sqrt{\lambda_1} & 0 \\[4pt]
0 & \sqrt{\lambda_2}
\end{bmatrix}
\]
\subsubsection{Polar Decomposition Theorem}

Let $\underline{F}$ be an invertible tensor (i.e.\ $\det \underline{F} > 0$).  
Then there exist unique positive definite, symmetric tensors 
$\underline{U}$ and $\underline{V}$, and a rotation tensor $\underline{R}$ such that:
\begin{align}
\underline{F} = \underline{R}\underline{U} = \underline{V}\underline{R}
\end{align}

Each of these decompositions is unique.

\begin{itemize}
    \item $\underline{F} = \underline{R}\underline{U}$ is the \textbf{right polar decomposition} of $\underline{F}$
    \item $\underline{F} = \underline{V}\underline{R}$ is the \textbf{left polar decomposition} of $\underline{F}$
\end{itemize}

Here:
\[
\underline{R} \text{ is a rotation tensor (orthogonal, } \underline{R}^T\underline{R} = \underline{I},\ \det \underline{R} > 0)
\]
\[
\underline{U} \text{ is the right stretch tensor,} 
\qquad \underline{V} \text{ is the left stretch tensor.}
\]

---

\textbf{Relationships between the tensors:}
\begin{align}
\underline{C} &= \underline{F}^T \underline{F} = \underline{U}^2 \\[4pt]
\underline{B} &= \underline{F}\underline{F}^T = \underline{V}^2
\end{align}

Thus,
\begin{align}
\underline{U} = \sqrt{\underline{C}},
\qquad
\underline{V} = \sqrt{\underline{B}}
\end{align}

---

\begin{proof}
Let $\lambda_i > 0$ be the eigenvalues of $\underline{C}$ and 
$\underline{N}^{(i)}$ the corresponding eigenvectors.  
Since $\underline{C}$ is symmetric and positive definite,
\[
\underline{C} = \sum_{i=1}^3 \lambda_i\, \underline{N}^{(i)} \otimes \underline{N}^{(i)}
\]
Define
\[
\underline{U} = \sum_{i=1}^3 \sqrt{\lambda_i}\, \underline{N}^{(i)} \otimes \underline{N}^{(i)}
\]
so that $\underline{U}$ is symmetric and positive definite.

Now define
\[
\underline{R} = \underline{F}\underline{U}^{-1}
\]

Then,
\begin{align}
\underline{R}^T\underline{R}
&= (\underline{U}^{-1})^T \underline{F}^T \underline{F} \underline{U}^{-1} \\[4pt]
&= \underline{U}^{-1} \underline{C} \underline{U}^{-1} \\[4pt]
&= \underline{U}^{-1} \underline{U}^2 \underline{U}^{-1} \\[4pt]
&= \underline{I}
\end{align}
Hence, $\underline{R}$ is orthogonal.

Furthermore, since $\det\underline{F} > 0$ and $\det\underline{U} > 0$,
\[
\det\underline{R} = \frac{\det\underline{F}}{\det\underline{U}} > 0
\]
so $\underline{R}$ is a \textbf{proper orthogonal} tensor.

---

Finally, defining
\[
\underline{V} = \underline{R}\underline{U}\underline{R}^T
\]
gives the left polar form:
\[
\underline{F} = \underline{V}\underline{R}
\]

---

\textbf{Summary:}
\[
\boxed{
\underline{F} = \underline{R}\underline{U} = \underline{V}\underline{R}
}
\quad \text{(Polar decomposition)}
\]
\[
\underline{U} = \sqrt{\underline{F}^T\underline{F}}, 
\qquad
\underline{V} = \sqrt{\underline{F}\underline{F}^T},
\qquad
\det \underline{R} = +1
\]
\end{proof}

\subsubsection{Geometric Interpretation of $\underline{U}$ and $\underline{R}$}

\textbf{Right stretch tensor $\underline{U}$:}

Using the spectral decomposition,
\begin{align}
\underline{U} 
= \sum_{i=1}^{3} \lambda_i \, \underline{N}^{(i)} \otimes \underline{N}^{(i)}
\end{align}

When $\underline{U}$ acts on a line element $d\underline{X}$,
\begin{align}
\underline{U}\, d\underline{X}
&= \sum_{i=1}^{3} \lambda_i 
   \underline{N}^{(i)} (\underline{N}^{(i)} \cdot d\underline{X}) \\[4pt]
&= \sum_{i=1}^{3} \lambda_i\, dX_i\, \underline{N}^{(i)}
\end{align}

Hence, the result of $\underline{U}\, d\underline{X}$ 
is an \textbf{extension of} $d\underline{X}$ by an amount $\lambda_i$ 
in the principal directions $\underline{N}^{(i)}$.

Thus, $\underline{U}$ represents a pure stretch 
along the principal directions of deformation.

---

\textbf{Rotation tensor $\underline{R}$:}

The rotation tensor $\underline{R}$ admits a similar geometric interpretation:
it rotates the vector $d\underline{X}$ by an angle $\theta$ about an axis 
through the origin in the direction of the eigenvector of $\underline{R}$ 
associated with the eigenvalue $\lambda = 1$.

\[
\text{Axis of rotation: eigenvector } \underline{n}
\quad \text{such that} \quad
\underline{R}\underline{n} = \underline{n}
\]

\textbf{Trace relation:}
\begin{align}
\operatorname{tr}[\underline{R}] = 2\cos\theta + 1
\end{align}

\subsubsection{Left Cauchy–Green Deformation Tensor}

The left Cauchy–Green deformation tensor connects back to the \textbf{current configuration}.

\begin{align}
d\underline{x} &= \underline{F} \, d\underline{X} \\[4pt]
ds^2 &= d\underline{x}\cdot d\underline{x}
= ( \underline{F} \, d\underline{X} ) \cdot ( \underline{F} \, d\underline{X} )
= d\underline{X} \cdot (\underline{F}^T \underline{F}) \, d\underline{X}
\end{align}

Define:
\begin{align}
\underline{B} = \underline{F}\underline{F}^T
\end{align}
so that
\begin{align}
ds^2 = d\underline{x}\cdot d\underline{x} 
= d\underline{X}\cdot \underline{C}\, d\underline{X},
\quad
d\underline{x}\cdot d\underline{x} 
= d\underline{x}\cdot (\underline{B}\, d\underline{x})
\end{align}

Thus, $\underline{C} = \underline{F}^T \underline{F}$ (right Cauchy–Green tensor)  
and $\underline{B} = \underline{F}\underline{F}^T$ (left Cauchy–Green tensor).

---

\textbf{Relation between $\underline{B}$ and $\underline{V}$:}
\begin{align}
\underline{B} = \underline{V}^2
\end{align}

Since both $\underline{C}$ and $\underline{B}$ are symmetric, positive definite tensors,
they each possess three eigenvalues and three corresponding orthogonal eigenvectors.

Let $\lambda_i^2$ be the eigenvalues of $\underline{C}$ with eigenvectors $\underline{N}^{(i)}$, and  
$\mu_i^2$ be the eigenvalues of $\underline{B}$ with eigenvectors $\underline{n}^{(i)}$.

---

\textbf{Connection between right and left polar decompositions:}
\begin{align}
\underline{F} &= \underline{R}\underline{U} = \underline{V}\underline{R}
\end{align}

Hence,
\begin{align}
\underline{R}\underline{U}\underline{R}^T &= \underline{V}
\end{align}
which shows that $\underline{U}$ and $\underline{V}$ share the same principal stretches $\lambda_i$,
but are expressed in different configurations.

\[
\underline{U} \leftrightarrow \text{reference configuration},
\qquad
\underline{V} \leftrightarrow \text{current configuration.}
\]

---

\subsubsection{Eigenvalue Relations}

If 
\[
\underline{C}\underline{N}^{(i)} = \lambda_i^2 \underline{N}^{(i)}
\]
then
\[
\underline{B}\underline{n}^{(i)} = \lambda_i^2 \underline{n}^{(i)}
\]
and
\[
\underline{n}^{(i)} = \underline{R}\,\underline{N}^{(i)}.
\]

\textbf{Proof:}
\begin{align}
\underline{B}\underline{n}^{(i)} 
&= \underline{F}\underline{F}^T (\underline{R}\underline{N}^{(i)}) \\[4pt]
&= (\underline{R}\underline{U})(\underline{R}\underline{U})^T \underline{R}\underline{N}^{(i)} \\[4pt]
&= \underline{R}\underline{U}\underline{U}^T\underline{N}^{(i)} \\[4pt]
&= \underline{R}\underline{U}^2\underline{N}^{(i)} \\[4pt]
&= \lambda_i^2 (\underline{R}\underline{N}^{(i)}) = \lambda_i^2 \underline{n}^{(i)}
\end{align}

Thus, $\underline{n}^{(i)} = \underline{R}\underline{N}^{(i)}$ is an eigenvector of $\underline{B}$
corresponding to the eigenvalue $\lambda_i^2$.

---

\subsubsection{Principal Directions of $\underline{C}$ and $\underline{U}$}

\begin{itemize}
    \item Under the \textbf{right polar decomposition} ($\underline{F} = \underline{R}\underline{U}$):  
    $\underline{U}$ stretches $\underline{N}^{(i)}$ by $\lambda_i$, 
    then $\underline{R}$ rotates the stretched vectors to $\underline{n}^{(i)}$.

    \item Under the \textbf{left polar decomposition} ($\underline{F} = \underline{V}\underline{R}$):  
    $\underline{R}$ first rotates $\underline{N}^{(i)}$ to $\underline{n}^{(i)}$, 
    then $\underline{V}$ stretches $\underline{n}^{(i)}$ by $\lambda_i$.
\end{itemize}

\[
\underline{F}\underline{N}^{(i)} = \lambda_i \underline{n}^{(i)}
\qquad
\underline{F}^T\underline{n}^{(i)} = \lambda_i \underline{N}^{(i)}
\]

---

\subsubsection{Summary Diagram}

\[
\text{Reference configuration: } \underline{C},\ \lambda_i^2,\ \underline{N}^{(i)}
\quad
\xrightarrow{\ \underline{F}\ }
\quad
\text{Current configuration: } \underline{B},\ \lambda_i^2,\ \underline{n}^{(i)}
\]

---

\subsection{General Strain Tensor}

A general class of strain measures $\underline{E}$ can be defined as a symmetric tensor whose principal directions coincide with those of $\underline{U}$, and whose eigenvalues are functions of $\lambda_i$:
\begin{align}
E_i = f(\lambda_i)
\end{align}

That is,
\[
\underline{E} = \sum_{i=1}^{3} E_i \, \underline{N}^{(i)} \otimes \underline{N}^{(i)}
\]

\textbf{Properties:}
\[
f(1) = 0, \qquad f'(1) = 1, \qquad f'(\lambda_i) > 0 \text{ for all } \lambda_i > 0
\]

Examples of $f(\lambda)$ include:
\begin{align*}
\text{Green–Lagrange strain:} &\quad f(\lambda) = \tfrac{1}{2}(\lambda^2 - 1) \\[4pt]
\text{Almansi strain:} &\quad f(\lambda) = \tfrac{1}{2}(1 - \lambda^{-2}) \\[4pt]
\text{Hencky (logarithmic) strain:} &\quad f(\lambda) = \ln(\lambda)
\end{align*}

\subsubsection{Green–Lagrange Strain Tensor}

The Green–Lagrange strain is defined as
\begin{align}
\underline{E} = \tfrac{1}{2}(\underline{C} - \underline{I})
= \tfrac{1}{2}(\underline{F}^T \underline{F} - \underline{I})
\end{align}

In indicial notation:
\begin{align}
E_{ij} 
&= \tfrac{1}{2}(C_{ij} - \delta_{ij}) 
= \tfrac{1}{2}(F_{ki}F_{kj} - \delta_{ij})
\end{align}

Let
\[
x_i = X_i + u_i
\quad \Rightarrow \quad
\frac{\partial x_i}{\partial X_j}
= \delta_{ij} + \frac{\partial u_i}{\partial X_j}
\]
so that
\begin{align}
E_{ij}
&= \tfrac{1}{2}\Big[ 
\frac{\partial u_i}{\partial X_j} 
+ \frac{\partial u_j}{\partial X_i} 
+ \frac{\partial u_k}{\partial X_i} 
  \frac{\partial u_k}{\partial X_j}
\Big]
\end{align}

\textbf{Interpretation:}  
The Green–Lagrange strain consists of linear (symmetric part of the displacement gradient) and nonlinear (quadratic) terms.  
Different equivalent forms:
\[
E_{ij} = \tfrac{1}{2}(u_{i,j} + u_{j,i} + u_{k,i}u_{k,j})
\]

---

\subsubsection{Rigid Body Motion}

For a rigid body motion (RBM):
\[
\underline{F} = \underline{R}
\quad \Rightarrow \quad
\underline{C} = \underline{R}^T \underline{R} = \underline{I}
\]
\[
\underline{E} = \tfrac{1}{2}(\underline{C} - \underline{I}) = 0
\]
Hence, $\underline{E}$ vanishes under RBM, making it a good strain measure.

---

\subsubsection{Infinitesimal (Small) Strain Approximation}

Define the infinitesimal strain tensor:
\begin{align}
\underline{\varepsilon} 
= \tfrac{1}{2}(\nabla \underline{u} + \nabla \underline{u}^T)
\end{align}

In components:
\[
\varepsilon_{ij} = \tfrac{1}{2}\left(
\frac{\partial u_i}{\partial x_j} 
+ \frac{\partial u_j}{\partial x_i}
\right)
\]

---

\textbf{Relation to Green–Lagrange strain:}
\[
\underline{E} 
= \tfrac{1}{2}\big[(\nabla \underline{u}) + (\nabla \underline{u})^T + (\nabla \underline{u})^T (\nabla \underline{u})\big]
\]
When $\|\nabla \underline{u}\| \ll 1$, the quadratic term is negligible, and
\[
\underline{E} \approx \underline{\varepsilon}
\]
Thus, the infinitesimal strain is the linearized form of $\underline{E}$ valid for small deformations.

---

\subsubsection{Almansi (Eulerian) Strain Tensor}

The Almansi strain is defined with respect to the \textbf{current configuration:}
\begin{align}
ds^2 - ds_0^2 = 2\, d\underline{x}\cdot \underline{e}\, d\underline{x}
\end{align}

Since $d\underline{x} = \underline{F}\, d\underline{X}$,
\begin{align}
\underline{e} 
= \tfrac{1}{2}(\underline{I} - \underline{B}^{-1})
= \tfrac{1}{2}(\underline{I} - \underline{F}^{-T} \underline{F}^{-1})
\end{align}

In index form:
\begin{align}
e_{ij} 
= \tfrac{1}{2}\left(\delta_{ij} - (B^{-1})_{ij}\right)
= \tfrac{1}{2}\left(\delta_{ij} - F^{-1}_{ki} F^{-1}_{kj}\right)
\end{align}

---

\textbf{Relation to Principal Directions:}
The Almansi strain $\underline{e}$ shares the same principal directions as the left stretch tensor $\underline{V}$:
\[
\underline{B} = \underline{V}^2
\quad \Rightarrow \quad
\underline{e} 
= \tfrac{1}{2}(\underline{I} - \underline{V}^{-2})
\]
and
\[
e_i = \tfrac{1}{2}(1 - \lambda_i^{-2})
\]

Thus, $\underline{E}$ and $\underline{e}$ have identical principal directions, 
but refer to different configurations:
\[
\underline{E} \rightarrow \text{Reference (Lagrangian)}, 
\quad
\underline{e} \rightarrow \text{Current (Eulerian)}
\]

\subsubsection{Example: Simple Shear}

Parallel planes are displaced relative to each other by an amount proportional to their distance apart.

\[
\tan\gamma = K
\]


\textbf{Equations of Motion:}
\[
x_1 = X_1 + K X_2, 
\quad x_2 = X_2, 
\quad x_3 = X_3
\]

\textbf{Deformation Gradient:}
\[
[F] =
\begin{bmatrix}
1 & K & 0 \\
0 & 1 & 0 \\
0 & 0 & 1
\end{bmatrix}
\]
Only a function of $K = \tan \gamma$.

\textbf{Jacobian:}
\[
J = \det F = 1 \quad \Rightarrow \quad \text{No volume change (incompressible)}
\]

\textbf{Right Cauchy–Green Tensor:}
\[
[C] = F^T F =
\begin{bmatrix}
1 & K & 0 \\
K & K^2 + 1 & 0 \\
0 & 0 & 1
\end{bmatrix}
\]

\textbf{Green–Lagrange Strain:}
\[
[E] = \tfrac{1}{2}(C - I)
= \tfrac{1}{2}
\begin{bmatrix}
0 & K & 0 \\
K & K^2 & 0 \\
0 & 0 & 0
\end{bmatrix}
\]
---

\subsubsection{Extension}

Consider a triaxial stretch:
\[
x_1 = \lambda_1 X_1, \quad
x_2 = \lambda_2 X_2, \quad
x_3 = \lambda_3 X_3
\]

\textbf{Deformation Gradient:}
\[
[F] =
\begin{bmatrix}
\lambda_1 & 0 & 0 \\
0 & \lambda_2 & 0 \\
0 & 0 & \lambda_3
\end{bmatrix}
\]

\textbf{Green Strain:}
\[
[E] = \tfrac{1}{2}(C - I)
= \tfrac{1}{2}
\begin{bmatrix}
\lambda_1^2 - 1 & 0 & 0 \\
0 & \lambda_2^2 - 1 & 0 \\
0 & 0 & \lambda_3^2 - 1
\end{bmatrix}
\]

\textbf{Special Cases:}
\begin{itemize}
\item $\lambda_1 = \lambda_2 = \lambda_3 \Rightarrow$ uniform dilation.
\item If $\lambda_1 \lambda_2 \lambda_3 = 1$, the deformation is volume-preserving.
\end{itemize}

---

\subsubsection{Small Deformation (Infinitesimal Strain)}

For small displacements:
\[
x_i = X_i + u_i(X, t)
\]
then
\[
\frac{\partial x_i}{\partial X_j}
= \delta_{ij} + \frac{\partial u_i}{\partial X_j}
\]

The infinitesimal strain tensor is defined as
\[
\varepsilon_{ij} = \tfrac{1}{2}\left(
\frac{\partial u_i}{\partial X_j} + \frac{\partial u_j}{\partial X_i}
\right)
\]

If $\|\nabla u\| \ll 1$, quadratic terms are negligible, and
\[
E_{ij} \approx \varepsilon_{ij}
\]

---

\subsubsection{Geometric Interpretation of Infinitesimal Strain}

For infinitesimal strain ($E \approx \varepsilon$):
\[
ds^2 - ds_0^2 = 2\, d\underline{X}\cdot \underline{\varepsilon}\, d\underline{X}
\]

Let $d\underline{X} = dS_0\, \underline{N}$, then
\[
ds^2 = ds_0^2 + 2\, ds_0^2\, \underline{N}\cdot \underline{\varepsilon}\, \underline{N}
\]

Thus, the relative change in length is
\[
\frac{ds - ds_0}{ds_0} = \underline{N}\cdot \underline{\varepsilon}\, \underline{N}
\]

For principal directions $\underline{N}^{(i)}$,
\[
\varepsilon_i = \underline{N}^{(i)}\cdot \underline{\varepsilon}\, \underline{N}^{(i)}
\]
represents the normal strain along that direction.

---

\subsubsection{Engineering Shear Strain}

Let $\underline{N}^{(1)}$ and $\underline{N}^{(2)}$ be initially orthogonal unit vectors.  
After deformation, the angle between the lines changes by $\gamma$.

\[
\cos(\tfrac{\pi}{2} - \gamma)
= 2\, \underline{N}^{(1)}\cdot \underline{\varepsilon}\, \underline{N}^{(2)}
\]
\[
\Rightarrow \gamma = 2\, \varepsilon_{12}
\]

Thus,
\[
\boxed{\gamma = 2\, \varepsilon_{12}}
\quad \text{and} \quad
\varepsilon_{12} = \tfrac{1}{2}\gamma
\]

---

\textbf{Remarks:}
\begin{itemize}
\item $\gamma$ is the \emph{engineering shear strain} — not a tensorial quantity.
\item $\varepsilon_{ij}$ is the \emph{tensorial shear strain} — a true component of the strain tensor.
\end{itemize}

\subsection{Infinitesimal Strain Tensor}

The infinitesimal strain tensor is defined as
\begin{align}
\varepsilon_{ij} = \tfrac{1}{2}\left( 
\frac{\partial u_i}{\partial x_j} 
+ \frac{\partial u_j}{\partial x_i}
\right)
\end{align}
It is symmetric by definition:
\[
\varepsilon_{ij} = \varepsilon_{ji}
\]

---

\subsubsection{Geometric Interpretation}

Consider two infinitesimal fibers $dX_1, dX_2$ along coordinate directions.  
If $u_i(X_1, X_2, X_3)$ are the displacement components, then for points $A, B, C$ on these fibers:

\[
u_B^i = u_A^i + \frac{\partial u_i}{\partial X_1} dX_1, 
\quad
u_C^i = u_A^i + \frac{\partial u_i}{\partial X_2} dX_2
\]

Elongation of fiber $AB$ of initial length $dX_1$:
\[
\frac{u_B^1 - u_A^1}{dX_1} = \frac{\partial u_1}{\partial X_1} = \varepsilon_{11}
\]
Similarly for fiber $AC$:
\[
\frac{u_C^2 - u_A^2}{dX_2} = \frac{\partial u_2}{\partial X_2} = \varepsilon_{22}
\]

Thus, $\varepsilon_{11}, \varepsilon_{22}, \varepsilon_{33}$ are the \textbf{normal strain components}.

---

\subsubsection{Shear Strain and Rotation of Fibers}

The fiber $AB$ is rotated by an angle $\beta$ given by
\[
\tan\beta = \frac{\partial u_2}{\partial X_1}
\]
Similarly, the fiber $AC$ is rotated by an angle $\gamma$ given by
\[
\tan\gamma = \frac{\partial u_1}{\partial X_2}
\]
The change in the originally right angle $BAC$ is
\[
\Delta\alpha = \beta + \gamma = 
\frac{\partial u_2}{\partial X_1} 
+ \frac{\partial u_1}{\partial X_2} 
= 2\,\varepsilon_{12}
\]
Hence, $\varepsilon_{12}$, $\varepsilon_{13}$, $\varepsilon_{23}$ are the \textbf{shear strain components.}

\[
\boxed{\gamma_{ij} = 2\, \varepsilon_{ij}}
\]

Angles $\beta$ and $\gamma$ are taken positive if the corresponding fiber rotates toward the positive coordinate direction.

---

\subsubsection{Infinitesimal Rotation Tensor}

We now define the infinitesimal rotation tensor:
\begin{align}
\omega_{ij} 
= \tfrac{1}{2}\left(
\frac{\partial u_i}{\partial x_j}
- \frac{\partial u_j}{\partial x_i}
\right)
\end{align}
It is antisymmetric:
\[
\omega_{ij} = -\omega_{ji}
\]

In matrix form:
\[
[\omega] =
\begin{bmatrix}
0 & -\omega_3 & \omega_2 \\
\omega_3 & 0 & -\omega_1 \\
-\omega_2 & \omega_1 & 0
\end{bmatrix}
\]

Each component corresponds to a rotation about a principal axis:
\[
\omega_1 \Rightarrow \text{rotation about } x_1, \quad
\omega_2 \Rightarrow \text{rotation about } x_2, \quad
\omega_3 \Rightarrow \text{rotation about } x_3
\]

---

\subsubsection{Decomposition of the Displacement Gradient}

Any displacement gradient can be decomposed as:
\[
\nabla \underline{u} 
= \underline{\varepsilon} + \underline{\omega}
\]
where $\underline{\varepsilon}$ is symmetric and $\underline{\omega}$ is antisymmetric.

Hence, any infinitesimal deformation consists of:
\begin{itemize}
\item A \textbf{pure strain} (shape change without rigid rotation)
\item A \textbf{rigid body rotation}
\end{itemize}

\[
\frac{\partial u_i}{\partial x_j}
= \varepsilon_{ij} + \omega_{ij}
\]

---

\subsubsection{Rigid Body Motion}

For rigid body motion:
\[
\underline{F} = \underline{R}, 
\quad \underline{C} = \underline{I}, 
\quad \underline{E} = 0
\]
and
\[
\underline{u} = \underline{\omega}\, \underline{X} + \underline{d}
\]

Thus, for small rotations ($|\nabla u| = O(\varepsilon)$),
\[
\underline{\varepsilon} = 0, \quad \underline{\omega} = \tfrac{1}{2}(\nabla u - \nabla u^T)
\]

---

\subsubsection{Axial Vector of the Infinitesimal Rotation Tensor}

The antisymmetric tensor $\omega_{ij}$ corresponds to an axial vector $\zeta_i$ defined as:
\[
\zeta_i = \tfrac{1}{2}\, \epsilon_{ijk}\, \omega_{jk}
\]
or equivalently:
\[
[\underline{\zeta}] = 
\begin{bmatrix}
\omega_{23} \\ 
\omega_{31} \\ 
\omega_{12}
\end{bmatrix}
\]

Hence:
\[
\boxed{\underline{\omega}\, \underline{X} = \underline{\zeta} \times \underline{X}}
\]
since
\[
(\underline{\zeta} \times \underline{X})_i 
= \epsilon_{ijk} \zeta_j X_k
= \omega_{ik} X_k
\]
showing that $\underline{\omega}$ generates the infinitesimal rotation of the body about $\underline{\zeta}$.

\subsection{Infinitesimal Rotation and Volume Change}

\subsubsection{Infinitesimal Rotation Field}

At any point $P$, for an infinitesimal displacement field, the infinitesimal rotation of the body in the neighborhood of $P$ is given by the axial vector $\boldsymbol{\zeta}$:
\[
\boldsymbol{\zeta} = \tfrac{1}{2} \, \nabla \times \mathbf{u}
\]

The relative displacement of a neighboring point $P'$ with respect to $P$ is:
\[
d\mathbf{u} = \mathbf{u}(P') - \mathbf{u}(P)
= (\nabla \mathbf{u})\, d\mathbf{X}
\]

Under small deformation, this can be written as:
\[
d\mathbf{u} = \boldsymbol{\omega} \, d\mathbf{X} 
= \boldsymbol{\zeta} \times d\mathbf{X}
\]

Hence, the relative displacement corresponds to the infinitesimal rotation of $d\mathbf{X}$ about the axis $\boldsymbol{\zeta}$.

The magnitude $|\boldsymbol{\zeta}|$ represents the infinitesimal angle of rotation.

---

\subsubsection{Angle of Rotation}

For small $|\boldsymbol{\zeta}|$,
\[
|d\mathbf{u}| = |\boldsymbol{\zeta} \times d\mathbf{X}|
= |\boldsymbol{\zeta}|\,|d\mathbf{X}|\,\sin\theta
\approx |\boldsymbol{\zeta}|\,|d\mathbf{X}|
\]
Thus, $|\boldsymbol{\zeta}|$ expresses the infinitesimal rotation angle.

It can be shown that:
\[
\boldsymbol{\zeta} = \alpha\, \mathbf{n}
\]
where $\alpha$ is the infinitesimal rotation angle and $\mathbf{n}$ is the unit eigenvector of the rotation tensor $\mathbf{R}$:
\[
\mathbf{R} = \mathbf{I} + \alpha\, [\mathbf{n}]_\times + O(\alpha^2)
\]

---

\subsection{Jacobian and Volume Change}

The infinitesimal volume element in the reference configuration is:
\[
dV_0 = dX_1\, dX_2\, dX_3
\]

After deformation:
\[
dV = \det(\mathbf{F})\, dV_0 = J\, dV_0
\]
where $J = \det \mathbf{F}$ is the \textbf{Jacobian of deformation.}

\textbf{Interpretation:}
$J$ gives the ratio of deformed to undeformed volume:
\[
J = \frac{dV}{dV_0}
\]

---

\subsubsection{Principal Directions}

Let the deformation gradient in the principal stretch directions be:
\[
\mathbf{F} = \mathbf{R}\, \mathbf{U}, 
\quad \text{so that} \quad \det \mathbf{F} = \det \mathbf{U}
\]
Then:
\[
J = \lambda_1\, \lambda_2\, \lambda_3
\]

Thus, the normalized volume change is:
\[
\frac{\Delta V}{V_0} = J - 1
\]

---

\subsubsection{Infinitesimal Volume Change}

For small deformations:
\[
\mathbf{F} = \mathbf{I} + \nabla \mathbf{u}
\]
\[
J = \det(\mathbf{I} + \nabla \mathbf{u})
\approx 1 + \nabla \cdot \mathbf{u} = 1 + \text{tr}(\varepsilon)
\]
Hence:
\[
\frac{\Delta V}{V_0} = \nabla \cdot \mathbf{u} = \varepsilon_{kk}
\]
where $\varepsilon_{kk}$ is the \textbf{volumetric strain.}

---

\subsubsection{Uniform Dilatation}

For a uniform expansion or contraction:
\[
\varepsilon_{ij} = e\, \delta_{ij}
\]
where $e$ is constant.

If $e > 0$: expansion,  
if $e < 0$: contraction.

\[
\varepsilon =
\begin{bmatrix}
e & 0 & 0 \\
0 & e & 0 \\
0 & 0 & e
\end{bmatrix}
\]

Then:
\[
\varepsilon_{kk} = 3e
\]
and the total volume change is:
\[
\frac{\Delta V}{V_0} = 3e
\]
The displacement field for uniform dilatation is:
\[
\mathbf{u} = e\, \mathbf{X}
\]

---

\textbf{Summary:}
\begin{itemize}
\item $\boldsymbol{\zeta} = \tfrac{1}{2} \nabla \times \mathbf{u}$ — infinitesimal rotation vector
\item $J = \det \mathbf{F}$ — Jacobian of deformation (volume ratio)
\item $\varepsilon_{kk} = \text{tr}(\varepsilon)$ — infinitesimal volumetric strain
\item For uniform strain, $\frac{\Delta V}{V_0} = 3e$
\end{itemize}

\subsubsection{Pure Shear under Small Strains (Isochoric Case)}

Let the displacement field be
\[
u_1 = \varepsilon X_2, 
\quad
u_2 = \varepsilon X_1,
\quad
u_3 = 0
\]

Then the strain tensor is:
\[
[\varepsilon] =
\begin{bmatrix}
0 & \varepsilon & 0 \\
\varepsilon & 0 & 0 \\
0 & 0 & 0
\end{bmatrix}
\]

This corresponds to a pure shear deformation in the $X_1$–$X_2$ plane.



The principal directions are rotated by $\theta = 45^\circ$, and the principal strains are $\pm \varepsilon$.

\[
[Q] =
\begin{bmatrix}
\cos\theta & \sin\theta & 0 \\
-\sin\theta & \cos\theta & 0 \\
0 & 0 & 1
\end{bmatrix},
\quad
[\varepsilon'] = Q [\varepsilon] Q^T =
\begin{bmatrix}
\varepsilon & 0 & 0 \\
0 & -\varepsilon & 0 \\
0 & 0 & 0
\end{bmatrix}
\]

The deformation is \textbf{isochoric} (volume-preserving):
\[
\Delta V/V_0 = (1+\varepsilon)(1-\varepsilon) - 1 = -\varepsilon^2 \approx 0
\]

---

\subsubsection{Field with Rotation Component}

Now consider:
\[
u_1 = \varepsilon X_2, 
\quad
u_2 = 0,
\quad
u_3 = 0
\]

Then:
\[
[\varepsilon] =
\begin{bmatrix}
0 & \tfrac{\varepsilon}{2} & 0 \\
\tfrac{\varepsilon}{2} & 0 & 0 \\
0 & 0 & 0
\end{bmatrix},
\quad
[\omega] =
\tfrac{1}{2}
\begin{bmatrix}
0 & -\varepsilon & 0 \\
\varepsilon & 0 & 0 \\
0 & 0 & 0
\end{bmatrix}
\]

The antisymmetric part corresponds to an infinitesimal rotation vector:
\[
\boldsymbol{\zeta} =
\begin{bmatrix}
0 \\ 0 \\ -\tfrac{\varepsilon}{2}
\end{bmatrix}
= -\tfrac{\varepsilon}{2}\, \mathbf{e}_3
\]

Hence, this deformation field represents not only shear but also a small rotation about the $x_3$-axis.

---

\subsubsection{Eliminating the Rotation (Obtaining Pure Shear)}

To remove rotation, superimpose an equal and opposite infinitesimal rotation:
\[
\mathbf{u}^{(\omega)} =
\begin{bmatrix}
-\varepsilon X_2 \\
0 \\
0
\end{bmatrix}
\]
which corresponds to
\[
[\omega] =
\begin{bmatrix}
0 & \varepsilon & 0 \\
-\varepsilon & 0 & 0 \\
0 & 0 & 0
\end{bmatrix}
\]

Superposing the two fields:
\[
\begin{aligned}
u_1 &= \varepsilon X_2 + (-\varepsilon X_2) = \varepsilon X_2,\\
u_2 &= \varepsilon X_1 + 0 = \varepsilon X_1,\\
u_3 &= 0
\end{aligned}
\]
restores pure shear with no net rotation.

---

\subsubsection{General Relation for Small Strain}

The infinitesimal strain tensor is the symmetric part of the displacement gradient:
\[
\varepsilon_{ij} = \tfrac{1}{2}
\left( 
\frac{\partial u_i}{\partial X_j}
+ 
\frac{\partial u_j}{\partial X_i}
\right)
\]
and the antisymmetric part gives infinitesimal rotation:
\[
\omega_{ij} = \tfrac{1}{2}
\left( 
\frac{\partial u_i}{\partial X_j}
-
\frac{\partial u_j}{\partial X_i}
\right)
\]

In general:
\[
\nabla \mathbf{u} = \boldsymbol{\varepsilon} + \boldsymbol{\omega}
\]
and for small deformations:
\[
\text{volumetric strain: } 
\varepsilon_v = \tfrac{1}{3}\, \text{tr}(\boldsymbol{\varepsilon}),
\quad
\text{deviatoric strain: }
\varepsilon_{ij}' = \varepsilon_{ij} - \tfrac{1}{3}\varepsilon_{kk}\,\delta_{ij}
\]

---

\textbf{Summary:}
\begin{itemize}
\item Pure shear → equal and opposite principal strains $\pm \varepsilon$ at $45^\circ$.
\item Isochoric deformation → $\Delta V / V_0 \approx 0$.
\item Deformation gradient decomposes into symmetric strain + antisymmetric rotation.
\item Rotation vector $\boldsymbol{\zeta}$ defines the local axis of infinitesimal rotation.
\end{itemize}

\subsection{Compatibility Equations}

For a continuous displacement field $u_i(X_j)$, the infinitesimal strains are given by:
\[
\varepsilon_{ij} = \tfrac{1}{2}\left(
\frac{\partial u_i}{\partial X_j} +
\frac{\partial u_j}{\partial X_i}
\right)
\]

Taking second derivatives:
\[
\varepsilon_{ij,kl} =
\tfrac{1}{2}\left(
u_{i,jkl} + u_{j,ikl}
\right)
\]

Since mixed partial derivatives commute
($u_{i,jkl} = u_{i,jlk}$, etc.),
certain relationships among the strain components must hold
for a single-valued displacement field to exist.  
These are the \textbf{compatibility equations.}

\[
\boxed{
\varepsilon_{ij,kl} + \varepsilon_{kl,ij} 
- \varepsilon_{ik,jl} - \varepsilon_{jl,ik} = 0
}
\]

They ensure that the strain field $\varepsilon_{ij}$ 
is derivable from a continuous and single-valued displacement field $u_i$.

---

\subsubsection{Independent Equations}

In three dimensions, there are six independent strain components
and six independent compatibility equations.
The rest are either identities or redundant due to symmetry of $\varepsilon_{ij}$.

\[
\begin{aligned}
(1)\;& \varepsilon_{11,22} + \varepsilon_{22,11} - 2\varepsilon_{12,12} = 0, \\
(2)\;& \varepsilon_{22,33} + \varepsilon_{33,22} - 2\varepsilon_{23,23} = 0, \\
(3)\;& \varepsilon_{33,11} + \varepsilon_{11,33} - 2\varepsilon_{13,13} = 0, \\
(4)\;& \varepsilon_{12,23} + \varepsilon_{23,12} - \varepsilon_{13,22} - \varepsilon_{22,13} = 0, \\
(5)\;& \varepsilon_{23,31} + \varepsilon_{31,23} - \varepsilon_{21,33} - \varepsilon_{33,21} = 0, \\
(6)\;& \varepsilon_{31,12} + \varepsilon_{12,31} - \varepsilon_{32,11} - \varepsilon_{11,32} = 0.
\end{aligned}
\]

These are necessary and sufficient to guarantee
the existence of a continuous displacement field.

---

\subsubsection{Geometric Interpretation}

If the strain field $\varepsilon_{ij}$ satisfies these equations,
then neighboring infinitesimal line elements deform consistently,
and no “gaps” or “overlaps” occur when integrating the strain field across the body.

If the equations are not satisfied,
then the strain field corresponds to an incompatible deformation
(i.e., one that cannot be produced by a single-valued $u_i$ field).

Compatibility thus ensures that deformations are geometrically consistent
throughout a \emph{simply connected} domain
(i.e., one without holes).

---

\subsubsection{Path-Independence and Single-Valuedness of $u_i$}

Let $\mathbf{P}(\bar{X}_1, \bar{X}_2, \bar{X}_3)$ be a reference point
where $u_i$ is specified.
For any other point $\mathbf{P}'(X_1, X_2, X_3)$, the displacement is obtained via
integration along a curve $\Gamma$ joining $\mathbf{P}$ to $\mathbf{P}'$:
\[
u_i = \bar{u}_i + \int_{\Gamma} u_{i,j}\, dX_j
\]
For $u_i$ to be single-valued and continuous,
the integral must depend only on the end points,
not on the path $\Gamma$.
That is, the integral must be \emph{exact}:
\[
\oint_{\Gamma} u_{i,j}\, dX_j = 0
\]
for any closed path $\Gamma$.

---

\subsubsection{Césaro Line Integral Formulation}

Writing $u_{i,j}$ in terms of $\varepsilon_{ij}$ and $\omega_{ij}$:
\[
u_{i,j} = \varepsilon_{ij} + \omega_{ij}
\]
then:
\[
du_i = u_{i,j}\, dX_j
= (\varepsilon_{ij} + \omega_{ij})\, dX_j
\]

For path independence, this differential must be \textbf{exact}.
Thus:
\[
\frac{\partial u_{i,j}}{\partial X_k}
= \frac{\partial u_{i,k}}{\partial X_j}
\Rightarrow
\varepsilon_{ij,kl} + \varepsilon_{kl,ij} 
- \varepsilon_{ik,jl} - \varepsilon_{jl,ik} = 0
\]
which recovers the compatibility condition.

---

\subsubsection{Physical Meaning}

The compatibility conditions express the requirement that
a given strain distribution corresponds to a possible (continuous) displacement field.

If the strain field fails these conditions,
then “residual stresses” or “incompatibility” exist —
as occurs in bodies with internal constraints or dislocations.

---
\section{Balance Laws}

\subsection{Fundamental Postulates}

\subsubsection{Density}
At every material point $\mathbf{x}$ and time $t$, there exists a scalar field
\[
\rho(\mathbf{x},t) > 0
\]
called the \textbf{mass density}, such that the mass of a region $D$ is given by
\[
m(D,t) = \int_D \rho(\mathbf{x},t)\, dV
\]
Physically, $\rho$ is the limiting ratio of the mass $\Delta m$ to the volume $\Delta V$:
\[
\rho = \lim_{\Delta V \to 0} \frac{\Delta m}{\Delta V}
\]

---

\subsubsection{Body Force}
There exists a vector field $\mathbf{b}(\mathbf{x},t)$, called the \textbf{body force per unit volume}, such that the total body force acting on $D$ is
\[
\int_D \mathbf{b}(\mathbf{x},t)\, dV
\]
Typical examples include gravitational and electromagnetic forces.

If $\mathbf{b} = \rho \mathbf{g}$, then
\[
[\mathbf{b}] = \frac{\text{force}}{\text{volume}} = \frac{mg}{V} = \rho \mathbf{g}
\]

---

\subsubsection{Traction (Surface Force)}

There exists a vector-valued function $\mathbf{T}(\mathbf{x},t,\mathbf{n})$, where $\mathbf{n}$ is the unit normal to the surface $\partial D$, called the \textbf{traction vector}, such that the total surface force acting on $D$ by the surrounding material is:
\[
\int_{\partial D} \mathbf{T}(\mathbf{x},t,\mathbf{n})\, dS
\]

The infinitesimal traction on a surface element $\Delta S$ is defined by:
\[
\mathbf{T} = \lim_{\Delta S \to 0} \frac{\Delta \mathbf{F}}{\Delta S}
\quad \text{with units } \left[ \frac{\text{N}}{\text{m}^2} \right]
\]
where $\Delta \mathbf{F}$ is the force exerted by the exterior material across $\Delta S$.


---

\subsection{Cauchy’s Theorem}

The \textbf{Cauchy stress principle} asserts that the traction on any plane with unit normal $\mathbf{n}$ passing through a point is a linear function of $\mathbf{n}$:
\[
\mathbf{T}(\mathbf{x},t,\mathbf{n}) = \boldsymbol{\sigma}(\mathbf{x},t)\, \mathbf{n}
\]
where $\boldsymbol{\sigma}$ is the \textbf{Cauchy stress tensor}.

This implies that the stress tensor fully characterizes the state of stress at a point, since it defines the traction vector for all possible orientations of $\mathbf{n}$.

---

\subsection{Localization Theorem}

Let $g(\mathbf{x})$ be a continuous scalar or vector function in a region $\mathcal{V}$, and let $D \subset \mathcal{V}$ be any subregion.  
If
\[
\int_D g(\mathbf{x})\, dV = 0 \quad \forall D \subset \mathcal{V}
\]
then
\[
g(\mathbf{x}) = 0 \quad \forall \mathbf{x} \in \mathcal{V}.
\]

---

\begin{proof}
Suppose $g(\mathbf{x}^*) \neq 0$ for some point $\mathbf{x}^* \in \mathcal{V}$.
By continuity, there exists a small ball $A$ centered at $\mathbf{x}^*$ such that $g(\mathbf{x}) > 0$ for all $\mathbf{x} \in A$.

Choose $D = A \cap \mathcal{V}$.
Then:
\[
\int_D g(\mathbf{x})\, dV > 0
\]
which contradicts the assumption that $\int_D g(\mathbf{x})\, dV = 0$ for all $D$.
Hence, $g(\mathbf{x}) = 0$ everywhere in $\mathcal{V}$.


\end{proof}


---

\textbf{Physical meaning:}
The localization theorem allows us to convert integral equations (e.g., balance of momentum over a control volume) into \emph{local differential equations} that hold pointwise throughout the continuum.

\subsection{Conservation of Mass}

\subsubsection{Lagrangian (Material) Formulation}

We follow a material domain $D_t$ that moves with the body through time:
\[
D_0 \xrightarrow[]{t=t_1} D_{t_1} \xrightarrow[]{t=t_2} D_{t_2}
\]

The total mass in a material domain is constant:
\[
\frac{d}{dt}\int_{D_t} \rho\, dV = 0
\]
Using the deformation gradient $\mathbf{F} = \frac{\partial \mathbf{x}}{\partial \mathbf{X}}$ and its determinant $J = \det \mathbf{F}$, we write
\[
\frac{d}{dt}\int_{D_0} \rho\, J\, dV_0 = 0
\]
which localizes to
\[
\frac{d}{dt}(\rho J) = 0
\quad \text{or equivalently} \quad
\rho J = \rho_0 = \text{constant.}
\]

Hence,
\[
\rho = \frac{\rho_0}{J}, \qquad J > 0.
\]

---

\subsubsection{Eulerian (Spatial) Formulation}

Consider a spatially fixed control volume $D$.
Conservation of mass requires that the rate of change of mass inside $D$
equals the net mass flux across its boundary $\partial D$:
\[
\frac{d}{dt} \int_D \rho\, dV + \int_{\partial D} \rho\, \mathbf{v} \cdot \mathbf{n}\, dS = 0
\]

Applying the divergence theorem:
\[
\int_D \left( \frac{\partial \rho}{\partial t} + \nabla \cdot (\rho \mathbf{v}) \right)\, dV = 0
\]
Since this must hold for any $D$, we localize to obtain the
\textbf{continuity equation}:
\[
\boxed{
\frac{\partial \rho}{\partial t} + \nabla \cdot (\rho \mathbf{v}) = 0
}
\]

Expanding the divergence term:
\[
\frac{\partial \rho}{\partial t} + \mathbf{v} \cdot \nabla \rho + \rho\, \nabla \cdot \mathbf{v} = 0
\]
or, using the material derivative $D\rho/Dt$:
\[
\frac{D\rho}{Dt} + \rho\, \nabla \cdot \mathbf{v} = 0
\]

---

\subsubsection{Incompressibility}

For an incompressible material:
\[
\frac{D\rho}{Dt} = 0
\quad \Rightarrow \quad
\rho = \text{constant}
\quad \Rightarrow \quad
\nabla \cdot \mathbf{v} = 0
\]

---

\subsubsection{Relation Between Lagrangian and Eulerian Forms}

From $\frac{d}{dt}(\rho J) = 0$, we obtain:
\[
\dot{\rho} J + \rho \dot{J} = 0
\quad \Rightarrow \quad
\frac{\dot{J}}{J} = -\frac{\dot{\rho}}{\rho}
\]

But since
\[
\dot{J} = J\, \nabla \cdot \mathbf{v}
\]
we have
\[
\frac{D\rho}{Dt} + \rho\, \nabla \cdot \mathbf{v} = 0
\]
which is precisely the Eulerian continuity equation.

---

\subsubsection{Reynolds Transport Theorem (RTT)}

For any scalar field $\phi(\mathbf{x},t)$ and a material volume $D_t$:
\[
\frac{d}{dt} \int_{D_t} \phi\, dV
= \int_{D_t} \left( \frac{\partial \phi}{\partial t} + \nabla \cdot (\phi \mathbf{v}) \right) dV
\]

\textbf{Proof:}
\[
\frac{d}{dt} \int_{D_t} \phi\, dV
= \int_{D_t} \frac{\partial \phi}{\partial t}\, dV
+ \int_{\partial D_t} \phi\, \mathbf{v} \cdot \mathbf{n}\, dS
\]
and by the divergence theorem:
\[
\int_{\partial D_t} \phi\, \mathbf{v} \cdot \mathbf{n}\, dS
= \int_{D_t} \nabla \cdot (\phi \mathbf{v})\, dV
\]
so
\[
\frac{d}{dt} \int_{D_t} \phi\, dV
= \int_{D_t} \left( \frac{\partial \phi}{\partial t} + \nabla \cdot (\phi \mathbf{v}) \right) dV
\]
as required.

---

\textbf{Special Case: Conservation of Mass}

Let $\phi = \rho$, then
\[
\frac{d}{dt} \int_{D_t} \rho\, dV = 0
\quad \Rightarrow \quad
\frac{\partial \rho}{\partial t} + \nabla \cdot (\rho \mathbf{v}) = 0
\]

\textbf{Summary:}
\begin{align*}
\text{Lagrangian: } & \frac{d}{dt}(\rho J) = 0, \\
\text{Eulerian: } & \frac{\partial \rho}{\partial t} + \nabla \cdot (\rho \mathbf{v}) = 0, \\
\text{Incompressible: } & \nabla \cdot \mathbf{v} = 0.
\end{align*}

\subsection{Laws of Motion}

We follow the \textbf{material volume} (Lagrangian description).

---

\subsubsection{Linear Momentum (Force Equilibrium)}

Let $\mathbf{p} = \rho \mathbf{v}$ denote the linear momentum density.
The total linear momentum of a material volume $D_t$ is
\[
\mathbf{P} = \int_{D_t} \rho \mathbf{v}\, dV
\]
Newton’s second law states that the rate of change of momentum equals the total external force acting on $D_t$:
\[
\frac{d}{dt} \int_{D_t} \rho \mathbf{v}\, dV = 
\int_{D_t} \rho \mathbf{b}\, dV + \int_{\partial D_t} \mathbf{T}\, dS
\]
where $\rho \mathbf{b}$ is the body force per unit volume and $\mathbf{T}$ is the traction on $\partial D_t$.

---

\subsubsection{Angular Momentum (Moment Equilibrium)}

The total angular momentum of $D_t$ is
\[
\mathbf{H} = \int_{D_t} \rho\, \mathbf{x} \times \mathbf{v}\, dV
\]
and the total external moment is
\[
\dot{\mathbf{H}} = \int_{D_t} \rho\, \mathbf{x} \times \mathbf{b}\, dV 
+ \int_{\partial D_t} \mathbf{x} \times \mathbf{T}\, dS
\]

---

\subsection{Global Conservation of Linear Momentum}

Applying the Reynolds Transport Theorem (RTT):
\[
\frac{d}{dt}\int_{D_t} \rho \mathbf{v}\, dV 
= \int_{D_t} \frac{\partial (\rho \mathbf{v})}{\partial t}\, dV
+ \int_{\partial D_t} \rho \mathbf{v}\, (\mathbf{v}\cdot \mathbf{n})\, dS
\]

Substituting Cauchy’s relation $\mathbf{T} = \boldsymbol{\sigma} \mathbf{n}$ gives
\[
\int_{D_t} \frac{\partial (\rho \mathbf{v})}{\partial t}\, dV
+ \int_{\partial D_t} \rho \mathbf{v}\, (\mathbf{v}\cdot \mathbf{n})\, dS
= \int_{\partial D_t} \boldsymbol{\sigma}\mathbf{n}\, dS + \int_{D_t} \rho \mathbf{b}\, dV
\]

---

\subsubsection{Divergence Theorem and Localization}

Using $\int_{\partial D_t} \boldsymbol{\sigma}\mathbf{n}\, dS = \int_{D_t} \nabla \cdot \boldsymbol{\sigma}\, dV$
and localizing (since this holds for any $D_t$):
\[
\boxed{
\rho\, \dot{\mathbf{v}} = \nabla \cdot \boldsymbol{\sigma} + \rho\, \mathbf{b}
}
\]
This is the \textbf{local form of the balance of linear momentum} (Cauchy’s first law of motion).

---

\subsubsection{Quasi-static Motion}

If accelerations are negligible ($\dot{\mathbf{v}} \approx 0$):
\[
\nabla \cdot \boldsymbol{\sigma} + \rho\, \mathbf{b} = 0
\]
This is the \textbf{equilibrium equation} used in statics.

---

\subsubsection{Global Conservation of Angular Momentum}

The balance of moments gives:
\[
\frac{d}{dt}\int_{D_t} \rho\, \mathbf{x} \times \mathbf{v}\, dV
= \int_{D_t} \rho\, \mathbf{x} \times \mathbf{b}\, dV 
+ \int_{\partial D_t} \mathbf{x} \times (\boldsymbol{\sigma}\mathbf{n})\, dS
\]

Applying the divergence theorem:
\[
\int_{D_t} \mathbf{x} \times (\nabla \cdot \boldsymbol{\sigma})\, dV
+ \int_{D_t} \boldsymbol{\sigma}^{T} : \nabla \mathbf{x}\, dV
= \int_{D_t} \rho\, \mathbf{x} \times \mathbf{b}\, dV
\]

Localizing leads to:
\[
\mathbf{x} \times (\nabla \cdot \boldsymbol{\sigma}) + \text{skew}(\boldsymbol{\sigma}) = 0
\]
which implies that the stress tensor is symmetric:
\[
\boxed{
\boldsymbol{\sigma} = \boldsymbol{\sigma}^{T}
}
\]

---

\subsubsection{Summary of Balance Laws}

\begin{align*}
\textbf{Mass:} \quad & \dot{\rho} + \rho\, \nabla \cdot \mathbf{v} = 0 \\
\textbf{Linear Momentum:} \quad & \rho\, \dot{\mathbf{v}} = \nabla \cdot \boldsymbol{\sigma} + \rho\, \mathbf{b} \\
\textbf{Angular Momentum:} \quad & \boldsymbol{\sigma} = \boldsymbol{\sigma}^{T}
\end{align*}

---

\subsubsection{Interpretation}

- $\nabla \cdot \boldsymbol{\sigma}$ : internal (surface) forces per unit volume  
- $\rho\, \mathbf{b}$ : external body forces per unit volume  
- $\rho\, \dot{\mathbf{v}}$ : inertial forces per unit volume  

\textbf{In static equilibrium:}
\[
\nabla \cdot \boldsymbol{\sigma} + \rho\, \mathbf{b} = 0
\]

\textbf{In dynamic motion:}
\[
\nabla \cdot \boldsymbol{\sigma} + \rho\, \mathbf{b} = \rho\, \dot{\mathbf{v}}
\]

\subsection{Principal and Shear Stresses}

---

\subsubsection{Example: Bending Stress}
For a beam in pure bending, the stress distribution is given by
\[
\sigma = \frac{M x_2}{I}
\]
and the stress tensor can be represented as
\[
\boldsymbol{\sigma} =
\begin{bmatrix}
\sigma_{11} & 0 & 0 \\
0 & 0 & 0 \\
0 & 0 & 0
\end{bmatrix}
\]
where $M$ is the bending moment and $I$ is the moment of inertia.

---

\subsubsection{Traction and Normal Stress}

The traction vector on a plane with unit normal $\mathbf{n}$ is
\[
\mathbf{T} = \boldsymbol{\sigma} \mathbf{n}
\]
The \textbf{normal stress} acting on this plane is
\[
\sigma_n = \mathbf{n} \cdot \boldsymbol{\sigma} \mathbf{n}
\]

The normal stress $\sigma_n$ varies with the orientation $\mathbf{n}$.
To find the \textbf{principal stresses}, we seek orientations for which $\mathbf{T}$ is parallel to $\mathbf{n}$, i.e.,
\[
\mathbf{T} = \sigma_n \mathbf{n}
\]
This implies
\[
\boldsymbol{\sigma} \mathbf{n} = \sigma_n \mathbf{n}
\]
which is an eigenvalue problem.

Hence, the \textbf{principal stresses} $\sigma_1, \sigma_2, \sigma_3$ are the eigenvalues of $\boldsymbol{\sigma}$,
and the corresponding \textbf{principal directions} $\mathbf{n}_1, \mathbf{n}_2, \mathbf{n}_3$
are the eigenvectors.

\[
\text{min } \sigma_1 \leq \sigma_2 \leq \text{max } \sigma_3
\]

---

\subsubsection{Equilibrium Equations}

From the local balance of linear momentum:
\[
\sigma_{ij,j} + b_i = 0
\]
In component form:
\[
\begin{aligned}
\sigma_{11,1} + \sigma_{12,2} + \sigma_{13,3} + b_1 &= 0, \\
\sigma_{21,1} + \sigma_{22,2} + \sigma_{23,3} + b_2 &= 0, \\
\sigma_{31,1} + \sigma_{32,2} + \sigma_{33,3} + b_3 &= 0
\end{aligned}
\]

---

\subsubsection{Stress Decomposition}

Any stress tensor can be decomposed into its deviatoric and hydrostatic parts:
\[
\boldsymbol{\sigma} = \boldsymbol{\sigma}' + \frac{1}{3}(\text{tr}\,\boldsymbol{\sigma}) \mathbf{I}
\]
where  
$\boldsymbol{\sigma}'$ : deviatoric (distortional) component (shear-related)  
$\frac{1}{3}(\text{tr}\,\boldsymbol{\sigma}) \mathbf{I}$ : hydrostatic (volumetric) component.

---

\subsection{Maximum Shear Stress}

Let $\sigma_1, \sigma_2, \sigma_3$ be the principal stresses, corresponding to principal directions $\mathbf{e}_1, \mathbf{e}_2, \mathbf{e}_3$.

The shear stress on a plane with unit normal $\mathbf{n}$ is defined as
\[
\tau = \|\mathbf{T} - (\mathbf{T}\cdot \mathbf{n})\mathbf{n}\|
\]
where $\mathbf{T} = \boldsymbol{\sigma}\mathbf{n}$.

Expanding:
\[
\tau^2 = \mathbf{T}\cdot\mathbf{T} - (\mathbf{T}\cdot\mathbf{n})^2
\]

Let $\boldsymbol{\sigma}$ be diagonal in the principal stress basis:
\[
\boldsymbol{\sigma} =
\begin{bmatrix}
\sigma_1 & 0 & 0 \\
0 & \sigma_2 & 0 \\
0 & 0 & \sigma_3
\end{bmatrix}
\]
and $\mathbf{n} = [n_1, n_2, n_3]^T$.

Then:
\[
\mathbf{T} = \boldsymbol{\sigma}\mathbf{n} =
\begin{bmatrix}
\sigma_1 n_1 \\
\sigma_2 n_2 \\
\sigma_3 n_3
\end{bmatrix}
\]
\[
\mathbf{T}\cdot\mathbf{T} = \sigma_1^2 n_1^2 + \sigma_2^2 n_2^2 + \sigma_3^2 n_3^2
\]
\[
\mathbf{T}\cdot\mathbf{n} = \sigma_1 n_1^2 + \sigma_2 n_2^2 + \sigma_3 n_3^2
\]

Hence,
\[
\tau^2 = \sigma_1^2 n_1^2 + \sigma_2^2 n_2^2 + \sigma_3^2 n_3^2
- (\sigma_1 n_1^2 + \sigma_2 n_2^2 + \sigma_3 n_3^2)^2
\]

To find the directions of maximum $\tau$, use Lagrange multipliers with the constraint $n_1^2 + n_2^2 + n_3^2 = 1$.
Solving yields the maximum shear stresses:
\[
\boxed{
\tau_{\max} =
\frac{1}{2} \max\{|\sigma_1 - \sigma_2|,\, |\sigma_2 - \sigma_3|,\, |\sigma_3 - \sigma_1|\}
}
\]

and the planes of maximum shear occur halfway between the principal directions (at $45^\circ$).

\[
\text{Example: } \tau_{\max} = \frac{1}{2}(\sigma_1 - \sigma_3)
\]
\[
\mathbf{n} = \frac{1}{\sqrt{2}}(\mathbf{e}_1 \pm \mathbf{e}_3)
\]

---



---

\subsubsection{Summary}

\begin{align*}
\text{Traction vector:} & \quad \mathbf{T} = \boldsymbol{\sigma}\mathbf{n} \\
\text{Normal stress:} & \quad \sigma_n = \mathbf{n}\cdot\boldsymbol{\sigma}\mathbf{n} \\
\text{Principal stresses:} & \quad \boldsymbol{\sigma}\mathbf{n} = \sigma_n\mathbf{n} \\
\text{Maximum shear:} & \quad \tau_{\max} = \frac{1}{2}(\sigma_1 - \sigma_3)
\end{align*}

\subsection{Alternative Stress Measures}

---

\subsubsection{Invariants of the Cauchy Stress Tensor}

For any second-order tensor $\boldsymbol{\sigma}$, its principal invariants are:
\[
I_1 = \text{tr}(\boldsymbol{\sigma}) = \sigma_{kk}, \quad
I_2 = \tfrac{1}{2}[(\text{tr}\,\boldsymbol{\sigma})^2 - \text{tr}(\boldsymbol{\sigma}^2)], \quad
I_3 = \det(\boldsymbol{\sigma})
\]
These invariants remain unchanged under coordinate transformation and are often used in constitutive modeling.

---

\subsection{Cauchy and Nominal Stress}

\[
\sigma = \frac{F}{A} \quad \Rightarrow \quad \text{Cauchy stress (current configuration)}
\]
\[
\sigma_{\text{nom}} = \frac{F}{A_0} \quad \Rightarrow \quad \text{Nominal stress (reference configuration)}
\]

Let:
- $dS_0$ denote an infinitesimal area element in the reference configuration with normal $\mathbf{N}$,  
- $dS$ denote the corresponding area element in the current configuration with normal $\mathbf{n}$.

Then:
\[
d\mathbf{F} = \mathbf{T}^0\, dS_0 = \mathbf{T}\, dS
\]
where  
$\mathbf{T}$ : traction in current configuration (Cauchy traction),  
$\mathbf{T}^0$ : traction in reference configuration.

The Cauchy stress $\boldsymbol{\sigma}$ is defined via the surface traction:
\[
d\mathbf{F} = \boldsymbol{\sigma}\mathbf{n}\, dS
\]
and represents force per unit \textit{current} area.

---

\subsection{First Piola–Kirchhoff (Nominal) Stress Tensor}

We now seek a tensor $\mathbf{P}$ (or $\mathbf{T}^0$) that relates traction to the unit area in the \textit{reference} configuration:
\[
d\mathbf{F} = \mathbf{T}^0\, dS_0, \qquad 
\mathbf{T}^0 = \mathbf{P}\, \mathbf{N}
\]

The local balance of linear momentum in the current configuration is
\[
\text{div}\, \boldsymbol{\sigma} + \rho \mathbf{b} = \rho \dot{\mathbf{v}}
\]
Applying the divergence theorem and transforming to the reference configuration gives:
\[
\int_{V_0} \left( \frac{\partial P_{iK}}{\partial X_K} + \rho_0 b_i - \rho_0 \dot{v}_i \right) dV_0 = 0
\]
Since this holds for any material volume $V_0$, the localized form is:
\[
\boxed{
\frac{\partial P_{iK}}{\partial X_K} + \rho_0 b_i = \rho_0 \dot{v}_i
}
\]

The nominal stress $\mathbf{P}$ is related to the Cauchy stress by
\[
\boxed{
\mathbf{P} = J\, \boldsymbol{\sigma}\, \mathbf{F}^{-T}
}
\]
where $J = \det(\mathbf{F})$ and $\mathbf{F}$ is the deformation gradient.

$\mathbf{P}$ is generally \textit{not symmetric}.

---

\subsubsection{Relation Between Area Elements}

The mapping between $dS_0$ and $dS$ is given by:
\[
\mathbf{n}\, dS = J\, \mathbf{F}^{-T}\mathbf{N}\, dS_0
\]
or equivalently:
\[
dS = J\, |\mathbf{F}^{-T}\mathbf{N}|
\]
which follows from Nanson’s relation:
\[
\boxed{
\mathbf{n}\, dS = J\, \mathbf{F}^{-T}\mathbf{N}\, dS_0
}
\]

---

\subsection{Second Piola–Kirchhoff Stress Tensor}

To define a symmetric stress measure in the reference configuration, we introduce:
\[
\boxed{
\mathbf{S} = \mathbf{F}^{-1}\mathbf{P}
}
\]
Substituting the relation for $\mathbf{P}$:
\[
\mathbf{S} = J\, \mathbf{F}^{-1} \boldsymbol{\sigma} \mathbf{F}^{-T}
\]
$\mathbf{S}$ is called the \textbf{Second Piola–Kirchhoff (2PK) stress tensor}, and it is always symmetric because $\boldsymbol{\sigma}$ is symmetric.

Thus:
\[
\boxed{
\mathbf{S} = \mathbf{S}^T
}
\]

---

\subsubsection{Physical Meaning}

- $\mathbf{P}$ (First Piola–Kirchhoff): maps forces in the current configuration to areas in the reference configuration (force per undeformed area).  
- $\mathbf{S}$ (Second Piola–Kirchhoff): maps forces and areas both to the reference configuration (true tensor, symmetric).  
- $\boldsymbol{\sigma}$ (Cauchy): maps both forces and areas in the current configuration (true physical stress).

---

\subsubsection{Transformation Summary}

\[
\boxed{
\begin{aligned}
\mathbf{P} &= J\, \boldsymbol{\sigma}\, \mathbf{F}^{-T} &\text{(First Piola–Kirchhoff stress)} \\
\mathbf{S} &= \mathbf{F}^{-1}\mathbf{P} = J\, \mathbf{F}^{-1}\boldsymbol{\sigma}\mathbf{F}^{-T} &\text{(Second Piola–Kirchhoff stress)} \\
\boldsymbol{\sigma} &= \frac{1}{J}\, \mathbf{F}\, \mathbf{S}\, \mathbf{F}^T &\text{(Cauchy stress in terms of 2PK)}
\end{aligned}
}
\]

---

\subsubsection{Notes}
- $\mathbf{P}$ and $\mathbf{S}$ are defined in the reference (material) configuration.  
- $\boldsymbol{\sigma}$ is defined in the current (spatial) configuration.  
- For small deformations: $\mathbf{F} \approx \mathbf{I}$, $J \approx 1$, hence $\mathbf{S} \approx \boldsymbol{\sigma}$.

---


\textbf{Diagram Legend:}
- $\mathbf{T}$ : traction vector in current configuration  
- $\mathbf{T}^0$ : traction vector in reference configuration  
- $d\mathbf{F}$ : differential force  
- $dS$, $dS_0$ : area elements  
- $\mathbf{F}$ : deformation gradient  
- $J = \det(\mathbf{F})$

\subsection{Energy Balance and Stress–Strain Power Conjugacy}

---

\subsubsection{Physical Meaning of Stress Transformations}

Recall:
\[
d\mathbf{F} = \boldsymbol{\sigma}\, \mathbf{n}\, dS = \mathbf{T}\, dS
\]

To map $d\mathbf{F}$ back to the reference configuration, we use $\mathbf{F}^{-T}$:
\[
d\mathbf{F} = \mathbf{F}\, \mathbf{N}\, dS_0
\]

Define a traction vector in the reference configuration:
\[
\mathbf{T}^0 = \mathbf{S}\, \mathbf{N}
\]
where $\mathbf{S}$ is the \textbf{Second Piola–Kirchhoff stress tensor}.

Then:
\[
\boxed{
\mathbf{S} = J\, \mathbf{F}^{-1} \boldsymbol{\sigma} \mathbf{F}^{-T}
}
\]
and the local equilibrium equation in the reference configuration is:
\[
\frac{\partial (S_{iK}F_{jK})}{\partial X_j} + \rho_0 b_i = \rho_0 a_i
\]
All three equilibrium forms (Cauchy, 1st PK, 2nd PK) are equivalent:
\[
\nabla_x \cdot \boldsymbol{\sigma} + \rho \mathbf{b} = \rho \dot{\mathbf{v}}
\quad \Leftrightarrow \quad
\nabla_X \cdot \mathbf{P} + \rho_0 \mathbf{b} = \rho_0 \dot{\mathbf{v}}
\quad \Leftrightarrow \quad
\nabla_X \cdot (\mathbf{S}\mathbf{F}^T) + \rho_0 \mathbf{b} = \rho_0 \dot{\mathbf{v}}
\]

---

\subsection{Rate of Work and Energy Balance}

The differential work done by internal and external forces is:
\[
\delta W = \boldsymbol{\sigma} : \delta \mathbf{d}
\]
The rate of work is the \textbf{power}:
\[
\mathcal{P} = \frac{dW}{dt} = \boldsymbol{\sigma} : \mathbf{D} = \sigma_{ij} D_{ij}
\]
where $\mathbf{D}$ is the rate-of-deformation tensor.

---

\subsubsection{Global Form}

For a material volume $V(t)$ with surface $S(t)$:
\[
\frac{d}{dt}\int_{V(t)} \frac{1}{2}\rho v_i v_i\, dV
= \int_{S(t)} T_i v_i\, dS + \int_{V(t)} \rho b_i v_i\, dV
\]
Substituting $T_i = \sigma_{ij} n_j$ and using the divergence theorem:
\[
\int_{V(t)} \sigma_{ij,j} v_i\, dV + \int_{V(t)} \sigma_{ij} v_{i,j}\, dV
+ \int_{V(t)} \rho b_i v_i\, dV = \int_{V(t)} \rho a_i v_i\, dV
\]
Using the balance of linear momentum $\sigma_{ij,j} + \rho b_i = \rho a_i$:
\[
\boxed{
\int_{V(t)} \sigma_{ij} v_{i,j}\, dV = 0
}
\]
which shows that only the symmetric part of the velocity gradient contributes to internal work.

---

\subsubsection{Velocity Gradient Decomposition}

Define the velocity gradient:
\[
\mathbf{L} = \nabla \mathbf{v}, \qquad L_{ij} = \frac{\partial v_i}{\partial x_j}
\]

Decompose $\mathbf{L}$ into symmetric and antisymmetric parts:
\[
\boxed{
\mathbf{L} = \mathbf{D} + \boldsymbol{\Omega}
}
\]
where
\[
\mathbf{D} = \frac{1}{2}(\mathbf{L} + \mathbf{L}^T), \quad
\boldsymbol{\Omega} = \frac{1}{2}(\mathbf{L} - \mathbf{L}^T)
\]

- $\mathbf{D}$ : \textbf{rate-of-deformation tensor} (symmetric)  
  → measures strain rate  
- $\boldsymbol{\Omega}$ : \textbf{spin tensor} (antisymmetric)  
  → measures rigid-body rotation rate

Since $\boldsymbol{\Omega}$ does no work:
\[
\boldsymbol{\sigma} : \boldsymbol{\Omega} = 0
\]
Thus, the stress power per unit volume is:
\[
\boxed{
\dot{w}_{\text{int}} = \boldsymbol{\sigma} : \mathbf{D}
}
\]

---

\subsection{Stress Power and Conjugate Strain Measures}

\begin{table}[h!]
  \centering
  \renewcommand{\arraystretch}{1.2}
  \setlength{\tabcolsep}{10pt}
  \begin{tabular}{c c c}
  \toprule
  \textbf{Stress Measure} & \textbf{Conjugate Strain Measure} & \textbf{Power per Reference Volume} \\
  \midrule
  $\boldsymbol{\sigma}$ (Cauchy) & $\mathbf{D}$ & $\boldsymbol{\sigma} : \mathbf{D}$ \\
  $\mathbf{P}$ (1st PK) & $\dot{\mathbf{F}}$ & $\mathbf{P} : \dot{\mathbf{F}}$ \\
  $\mathbf{S}$ (2nd PK) & $\dot{\mathbf{E}}$ & $\mathbf{S} : \dot{\mathbf{E}}$ \\
  \bottomrule
  \end{tabular}
  \caption{Stress and conjugate strain measures with their corresponding power per reference volume.}
  \label{tab:stress_measures}
  \end{table}
Here $\mathbf{E}$ is the Green–Lagrange strain tensor:
\[
\mathbf{E} = \tfrac{1}{2}(\mathbf{F}^T \mathbf{F} - \mathbf{I})
\]
and its material rate is:
\[
\dot{\mathbf{E}} = \mathbf{F}^T \mathbf{D} \mathbf{F}
\]

Thus, the energy equivalence between configurations is:
\[
\boxed{
\boldsymbol{\sigma} : \mathbf{D} = \frac{1}{J}\, \mathbf{P} : \dot{\mathbf{F}} = \frac{1}{J}\, \mathbf{S} : \dot{\mathbf{E}}
}
\]

---

\subsubsection{Summary of Energy Relations}

\begin{align*}
\text{Power input:} &\quad \dot{W} = \int_{V(t)} \boldsymbol{\sigma} : \mathbf{D}\, dV \\
\text{Stress power density:} &\quad p = \boldsymbol{\sigma} : \mathbf{D} \\
\text{Total rate of work (current volume):} &\quad P = \int_V \boldsymbol{\sigma} : \mathbf{D}\, dV \\
\text{Total rate of work (reference volume):} &\quad P_0 = \int_{V_0} \mathbf{S} : \dot{\mathbf{E}}\, dV_0
\end{align*}

---

\subsubsection{Interpretation}
- $\boldsymbol{\sigma} : \mathbf{D}$ represents the **instantaneous mechanical power density** in the current configuration.  
- Only $\mathbf{D}$ contributes to deformation work; $\boldsymbol{\Omega}$ corresponds to pure rotation.  
- The equality $\boldsymbol{\sigma} : \mathbf{D} = \frac{1}{J}\mathbf{S} : \dot{\mathbf{E}}$ ensures energetic consistency across all configurations.

---

---

\subsubsection{Compact Summary}

\[
\boxed{
\begin{aligned}
\text{Velocity Gradient:} &\quad \mathbf{L} = \nabla \mathbf{v} \\
\text{Rate of Deformation:} &\quad \mathbf{D} = \tfrac{1}{2}(\mathbf{L} + \mathbf{L}^T) \\
\text{Spin Tensor:} &\quad \boldsymbol{\Omega} = \tfrac{1}{2}(\mathbf{L} - \mathbf{L}^T) \\
\text{Stress Power:} &\quad \boldsymbol{\sigma} : \mathbf{D} = \frac{1}{J}\, \mathbf{S} : \dot{\mathbf{E}}
\end{aligned}
}
\]

\subsection{Rate of Deformation and Work Conjugacy}

---

\subsubsection{Kinematics: Time Rate of Deformation Gradient}

\[
F_{ij} = \frac{\partial x_i}{\partial X_j}
\]
Taking its material time derivative:
\[
\dot{F}_{ij} = \frac{d}{dt}\left(\frac{\partial x_i}{\partial X_j}\right)
= \frac{\partial \dot{x}_i}{\partial X_j}
= \frac{\partial v_i}{\partial X_j}
\]
Using the chain rule:
\[
\frac{\partial v_i}{\partial X_j} = \frac{\partial v_i}{\partial x_k}\frac{\partial x_k}{\partial X_j}
\]
Thus:
\[
\boxed{
\dot{\mathbf{F}} = \mathbf{L}\, \mathbf{F}
\quad \text{where } \mathbf{L} = \nabla_x \mathbf{v}
}
\]

---

\subsubsection{Stress Power in Different Configurations}

From the previous section:
\[
P_0 = J\, \boldsymbol{\sigma} : \mathbf{D}
\]
and since $\mathbf{P} = J \boldsymbol{\sigma} \mathbf{F}^{-T}$,
\[
P_0 = \mathbf{P} : \dot{\mathbf{F}}
\]

Similarly, since $\mathbf{S} = \mathbf{F}^{-1}\mathbf{P}$,
\[
\boxed{
P_0 = \mathbf{S} : \dot{\mathbf{E}}
}
\]
where $\mathbf{E}$ is the Green–Lagrange strain tensor.

Therefore, $\boldsymbol{\sigma}$ is the \textbf{work-rate conjugate} of the rate of deformation $\mathbf{D}$, and $\mathbf{S}$ is the conjugate of $\dot{\mathbf{E}}$.

\[
\boldsymbol{\sigma} : \mathbf{D} = \frac{1}{J}\mathbf{S} : \dot{\mathbf{E}}
\]

---

\subsubsection{Green–Lagrange Strain and its Rate}

\[
\mathbf{E} = \frac{1}{2}(\mathbf{F}^T\mathbf{F} - \mathbf{I})
\]
Taking its rate:
\[
\dot{\mathbf{E}} = \frac{1}{2}(\dot{\mathbf{F}}^T\mathbf{F} + \mathbf{F}^T\dot{\mathbf{F}})
= \text{sym}(\mathbf{F}^T\dot{\mathbf{F}})
= \text{sym}(\mathbf{F}^T\mathbf{L}\mathbf{F})
\]

Hence, the work conjugacy condition becomes:
\[
P_0 = \text{tr}(\mathbf{S}\dot{\mathbf{E}})
\]

---

\subsection{Mean Stress Theorem}

---

The mean (or average) stress tensor over a volume $V$ is defined as:
\[
\bar{\sigma}_{ij} = \frac{1}{V}\int_V \sigma_{ij}\, dV
\]

Starting from the equilibrium condition:
\[
\int_S T_i x_j\, dS + \int_V b_i x_j\, dV
= \int_V \sigma_{ij}\, dV
\]
Substitute $T_i = \sigma_{ik}n_k$ and apply the divergence theorem:
\[
\int_V (\sigma_{ij,j} + b_i)x_j\, dV + \int_V \sigma_{ij}\, dV = 0
\]
Using equilibrium ($\sigma_{ij,j} + b_i = 0$):
\[
\boxed{
\bar{\sigma}_{ij} = \frac{1}{V}\int_V \sigma_{ij}\, dV
}
\]
which verifies that the mean stress equals the volume average of the stress field.

---

\subsection{Linear Elasticity: Hooke’s Law}

---

Assume small deformations and isotropy:
\[
\varepsilon_{ij} = \frac{1+\nu}{E}\sigma_{ij} - \frac{\nu}{E}\sigma_{kk}\delta_{ij}
\]
This is the general form of \textbf{Hooke’s law}.

For uniaxial stress $\sigma = \sigma_{11}$:
\[
\varepsilon_{11} = \frac{\sigma}{E}, \qquad
\varepsilon_{22} = \varepsilon_{33} = -\nu\frac{\sigma}{E}
\]

The volumetric strain is:
\[
\varepsilon_v = \varepsilon_{11} + \varepsilon_{22} + \varepsilon_{33}
= \frac{\sigma}{E}(1 - 2\nu)
\]

---

\subsection{Axial Compression Example}

Consider a cylindrical specimen compressed by an axial load $P$ (no friction):
\[
\sigma_z = \frac{P}{A} = \frac{P}{\pi r^2}
\]
Axial and radial strains:
\[
\varepsilon_z = \frac{\sigma_z}{E}, \quad
\varepsilon_r = \varepsilon_\theta = -\nu \frac{\sigma_z}{E}
\]
The volume change is:
\[
\frac{\Delta V}{V} = \varepsilon_v = (1 - 2\nu)\frac{\sigma_z}{E}
\]
If end friction exists (so lateral strain is restricted), the axial stress distribution becomes nonuniform, with shear stresses developing at the contact interface.

The external work done is:
\[
\Delta V = \frac{1-2\nu}{E} \int_S (p\, \mathbf{n}\cdot\mathbf{x})\, dS
\approx \frac{1-2\nu}{E} P h
\]
for uniform pressure $p$ over cross-section $A$.

---

\subsection{Summary Table}

\begingroup
\setlength{\tabcolsep}{8pt}
\renewcommand{\arraystretch}{1.15}
\begin{center}
\begin{tabular}{@{} l l l @{}}
\toprule
\textbf{Concept} & \textbf{Expression} & \textbf{Interpretation} \\
\midrule
$\mathbf{L}$ & $\displaystyle \nabla_x \mathbf{v}$ & Velocity gradient \\[4pt]
$\mathbf{D}$ & $\displaystyle \tfrac{1}{2}(\mathbf{L} + \mathbf{L}^T)$ & Rate of deformation tensor \\[4pt]
$\boldsymbol{\Omega}$ & $\displaystyle \tfrac{1}{2}(\mathbf{L} - \mathbf{L}^T)$ & Spin (rigid-body rotation) \\[4pt]
$\dot{\mathbf{E}}$ & $\displaystyle \text{sym}(\mathbf{F}^T \dot{\mathbf{F}})$ & Green strain rate \\[4pt]
$\boldsymbol{\sigma} : \mathbf{D}$ & Instantaneous power density & Work rate per current volume \\[4pt]
$\mathbf{S} : \dot{\mathbf{E}}$ & Material power density & Work rate per reference volume \\[4pt]
$\displaystyle \varepsilon_{ij} = \frac{1+\nu}{E}\sigma_{ij} - \frac{\nu}{E}\sigma_{kk}\delta_{ij}$ & Hooke’s law & Linear elastic isotropy \\
\bottomrule
\end{tabular}
\end{center}
\endgroup

\subsection{Energy Balance for Elastic Solids}

---

\subsubsection{Energy Rate Balance (Current Configuration)}

The external power acting on a material volume $V(t)$ is:
\[
P_{\text{ext}} = \int_{\partial V} \mathbf{T} \cdot \mathbf{v}\, dS + \int_V \rho \mathbf{b} \cdot \mathbf{v}\, dV
\]
Applying the divergence theorem and using $\mathbf{T} = \boldsymbol{\sigma}\mathbf{n}$:
\[
P_{\text{ext}} = \int_V \nabla \cdot (\boldsymbol{\sigma}\mathbf{v})\, dV + \int_V \rho \mathbf{b} \cdot \mathbf{v}\, dV
\]
\[
P_{\text{ext}} = \int_V \boldsymbol{\sigma} : \nabla \mathbf{v}\, dV + \int_V (\nabla \cdot \boldsymbol{\sigma} + \rho \mathbf{b}) \cdot \mathbf{v}\, dV
\]
Using momentum balance $\nabla \cdot \boldsymbol{\sigma} + \rho \mathbf{b} = \rho \dot{\mathbf{v}}$:
\[
\boxed{
P_{\text{ext}} = \int_V \boldsymbol{\sigma} : \mathbf{D}\, dV + \int_V \rho \dot{\mathbf{v}} \cdot \mathbf{v}\, dV
}
\]

The rate of change of total energy per current unit volume is:
\[
\frac{d}{dt}\int_V \left( \frac{1}{2}\rho \mathbf{v}\cdot\mathbf{v} + \rho e \right) dV = P_{\text{ext}}
\]
where $e$ is the internal energy per unit mass.  
Thus, $\boldsymbol{\sigma}:\mathbf{D}$ is the **stress working rate per current unit volume**.

---

\subsubsection{Energy Rate Balance (Reference Configuration)}

Rewriting in the material configuration:
\[
P_{\text{ext}} = \int_{\partial V_0} \mathbf{T}^0 \cdot \mathbf{v}\, dS_0 + \int_{V_0} \rho_0 \mathbf{b}^0 \cdot \mathbf{v}\, dV_0
\]
Apply divergence theorem in the reference frame:
\[
P_{\text{ext}} = \int_{V_0} \mathbf{P} : \nabla_X \mathbf{v}\, dV_0 + \int_{V_0} (\nabla_X \cdot \mathbf{P} + \rho_0 \mathbf{b}^0) \cdot \mathbf{v}\, dV_0
\]
Using equilibrium $\nabla_X \cdot \mathbf{P} + \rho_0 \mathbf{b}^0 = \rho_0 \dot{\mathbf{v}}$:
\[
\boxed{
P_{\text{ext}} = \int_{V_0} \mathbf{P} : \dot{\mathbf{F}}\, dV_0 + \int_{V_0} \rho_0 \dot{\mathbf{v}} \cdot \mathbf{v}\, dV_0
}
\]
Thus, $\mathbf{P}:\dot{\mathbf{F}}$ is the **work rate conjugate in the reference configuration**.

---

\subsection{Strain Energy and Elastic Solids}

For elastic materials, the stress power is stored as recoverable strain energy per unit reference volume $w(\mathbf{E})$:
\[
\dot{w} = \mathbf{S} : \dot{\mathbf{E}}
\]
Integrating over time:
\[
w(\mathbf{E}) = \int_0^t \mathbf{S} : \dot{\mathbf{E}}\, dt
\]
Hence, for a hyperelastic (Green elastic) material,
\[
\boxed{
\mathbf{S} = \frac{\partial w}{\partial \mathbf{E}}
}
\]
and equivalently in the current configuration,
\[
\boldsymbol{\sigma} = \frac{1}{J} \mathbf{F} \frac{\partial w}{\partial \mathbf{E}} \mathbf{F}^T
\]

---

\subsection{Total Energy and Conservation}

The total energy of the body is:
\[
E(t) = K(t) + U(t)
\]
where
\[
K(t) = \int_V \frac{1}{2}\rho \mathbf{v}\cdot\mathbf{v}\, dV, \quad U(t) = \int_{V_0} w(\mathbf{E})\, dV_0
\]
If the body is at rest and unloaded at $t = 0$,
\[
\boxed{
E(t) = K(t) + U(t) = \text{constant}
}
\]
This expresses the **conservation of mechanical energy** in an elastic solid.

---

\subsection{Nonlinear Elastic Solids}

For general nonlinear elastic materials:
\[
\mathbf{P} = \frac{\partial w}{\partial \mathbf{F}}
\]
and
\[
\boxed{
\mathbf{S} = 2\,\frac{\partial w}{\partial \mathbf{C}}
\quad \text{where } \mathbf{C} = \mathbf{F}^T\mathbf{F}
}
\]

The power conjugate relationships remain:
\[
\mathbf{P} : \dot{\mathbf{F}} = \mathbf{S} : \dot{\mathbf{E}} = \boldsymbol{\sigma} : \mathbf{D}
\]

---

\subsection{Governing Equations of Nonlinear Elasticity}

Let $\mathbf{x} = \mathbf{x}(\mathbf{X}, t)$ and $\mathbf{u} = \mathbf{x} - \mathbf{X}$.

Then:
\[
\begin{aligned}
\text{Kinematics:} & \quad \mathbf{F} = \frac{\partial \mathbf{x}}{\partial \mathbf{X}} \\
\text{Strain:} & \quad \mathbf{E} = \frac{1}{2}(\mathbf{F}^T\mathbf{F} - \mathbf{I}) \\
\text{Constitutive:} & \quad \mathbf{S} = \mathbf{S}(\mathbf{E}) = \frac{\partial w}{\partial \mathbf{E}} \\
\text{Equilibrium:} & \quad \nabla_X \cdot \mathbf{S} + \rho_0 \mathbf{b} = 0
\end{aligned}
\]

Hence, the nonlinear elasticity problem involves:
\[
\boxed{
\text{Unknowns: } 3 \text{ displacements},\ 6 \text{ strains},\ 6 \text{ stresses} \quad \Rightarrow 15 \text{ total equations}
}
\]

---

\subsection{Summary of Hyperelasticity}

\begingroup
\setlength{\tabcolsep}{8pt}
\renewcommand{\arraystretch}{1.15}
\begin{center}
\begin{tabular}{@{} l l l @{}}
\toprule
\textbf{Variable} & \textbf{Definition} & \textbf{Role} \\
\midrule
$\mathbf{F}$ & $\displaystyle \frac{\partial \mathbf{x}}{\partial \mathbf{X}}$ & Deformation gradient \\[4pt]
$\mathbf{E}$ & $\displaystyle \tfrac{1}{2}(\mathbf{F}^T \mathbf{F} - \mathbf{I})$ & Green–Lagrange strain \\[4pt]
$\mathbf{S}$ & $\displaystyle 2\,\frac{\partial w}{\partial \mathbf{C}}$ & 2nd Piola–Kirchhoff stress \\[4pt]
$\mathbf{P}$ & $\displaystyle \frac{\partial w}{\partial \mathbf{F}}$ & 1st Piola–Kirchhoff stress \\[4pt]
$\boldsymbol{\sigma}$ & $\displaystyle \frac{1}{J}\,\mathbf{F}\mathbf{S}\mathbf{F}^T$ & Cauchy stress \\[4pt]
$w(\mathbf{E})$ & Strain energy density & Constitutive potential \\[4pt]
$\mathbf{S}:\dot{\mathbf{E}}$ & Stress power per reference volume & Energetic conjugate pair \\
\bottomrule
\end{tabular}
\end{center}
\endgroup




\subsection{Nonlinear Elastic Solids and Hyperelasticity}

---

\subsubsection{Definition of Hyperelasticity}

For a hyperelastic solid, the stress derives from a scalar potential—the strain energy density \( W \):
\[
\boxed{
t_{ij} F_{ij} = \frac{dW}{dt}
}
\]
Since this must hold for arbitrary rates \( \dot{F}_{ij} \), we have
\[
\boxed{
t_{ij} = \frac{\partial W}{\partial F_{ij}}
}
\]
This means that the stress is derived directly from the energy function.

---

\subsubsection{Frame Indifference (Objectivity)}

The strain energy density \( W \) must not change if a **rigid body motion** is superposed on a deformation:
\[
W(\mathbf{F}) = W(\mathbf{QF})
\quad \forall \text{ proper orthogonal } \mathbf{Q}, \text{ where } \mathbf{Q}^T\mathbf{Q} = \mathbf{I}, \ \det\mathbf{Q}=+1
\]

---

\subsection{Material Frame Indifference}

Consider the motion
\[
\mathbf{x} = \mathbf{x}(\mathbf{X},t)
\quad \text{with deformation gradient} \quad
\mathbf{F} = \frac{\partial \mathbf{x}}{\partial \mathbf{X}}.
\]
Let the stress and strain energy density depend on \(\mathbf{F}\):
\[
\mathbf{t} = \mathbf{t}(\mathbf{F}), \quad W = W(\mathbf{F}).
\]
Now, consider an equivalent motion obtained by superposing a rigid rotation:
\[
\tilde{\mathbf{x}} = \mathbf{Q}(t)\mathbf{x} + \mathbf{c}(t),
\]
where \(\mathbf{Q}\) is an orthogonal rotation tensor.

Then:
\[
\tilde{\mathbf{F}} = \frac{\partial \tilde{\mathbf{x}}}{\partial \mathbf{X}} = \mathbf{QF}.
\]
Objectivity requires that both motions be described by the same energy function:
\[
\boxed{
W(\mathbf{F}) = W(\mathbf{QF}) \quad \forall \mathbf{Q}
}
\]

---

\subsection{Consequence: Dependence on \(\mathbf{C} = \mathbf{F}^T\mathbf{F}\)}

Consider the polar decomposition:
\[
\mathbf{F} = \mathbf{R}\mathbf{U},
\]
where \(\mathbf{R}\) is a rotation tensor (\(\mathbf{R}^T\mathbf{R} = \mathbf{I}\)) and \(\mathbf{U}\) is the right stretch tensor.

Since \(W\) is invariant under superposed rigid rotations:
\[
W(\mathbf{F}) = W(\mathbf{R}\mathbf{U}) = W(\mathbf{U})
\]
Thus, \(W\) can depend only on symmetric measures of strain. Because \(\mathbf{C} = \mathbf{F}^T\mathbf{F} = \mathbf{U}^2\):
\[
\boxed{
W = W(\mathbf{C})
}
\]

---

\subsection{Derivation of Stress from \(W(\mathbf{C})\)}

We have:
\[
t_{ij} = \frac{\partial W}{\partial F_{ij}}
= \frac{\partial W}{\partial C_{pq}} \frac{\partial C_{pq}}{\partial F_{ij}}
\]
and since
\[
C_{pq} = F_{rp}F_{rq},
\quad \frac{\partial C_{pq}}{\partial F_{ij}} = \delta_{pi}F_{jq} + F_{ip}\delta_{qj},
\]
we find:
\[
t_{ij} = 2 F_{ir} \frac{\partial W}{\partial C_{rj}}
\]
Therefore, defining the second Piola–Kirchhoff stress as:
\[
\boxed{
S_{rj} = 2\frac{\partial W}{\partial C_{rj}},
\quad \text{we obtain} \quad
t = \frac{1}{J}\mathbf{F}\mathbf{S}\mathbf{F}^T.
}
\]

---

\subsection{Symmetry and Work Conjugacy}

- \( \mathbf{S} \) is symmetric because \( \mathbf{C} \) is symmetric.  
- The stress power per unit reference volume is:
  \[
  P_0 = \mathbf{S} : \dot{\mathbf{E}} = \frac{dW}{dt}
  \]
- Therefore, \( \mathbf{S} \) is work-conjugate to the Green–Lagrange strain \( \mathbf{E} \).

---

\subsection{Infinitesimal Limit}

For small strains:
\[
\sigma_{ij} = \frac{\partial W}{\partial \varepsilon_{ij}}
\]
If \( W = \tfrac{1}{2}\sigma_{ij}\varepsilon_{ij} = \tfrac{1}{2}E\varepsilon^2 \),
then:
\[
\boxed{
\sigma = E\varepsilon
}
\]
which recovers Hooke’s law in the infinitesimal limit.

---

\subsection{Summary of Objectivity and Hyperelastic Constitutive Law}

\begingroup
\setlength{\tabcolsep}{8pt}
\renewcommand{\arraystretch}{1.15}
\begin{center}
\begin{tabular}{@{} l l l @{}}
\toprule
\textbf{Concept} & \textbf{Tensor / Expression} & \textbf{Meaning} \\
\midrule
Objectivity & $\displaystyle W(\mathbf{F}) = W(\mathbf{QF})$ & Energy unchanged by rigid rotation \\[4pt]
Consequence & $\displaystyle W = W(\mathbf{C})$ & Depends only on stretch invariants \\[4pt]
2nd PK Stress & $\displaystyle \mathbf{S} = 2\,\frac{\partial W}{\partial \mathbf{C}}$ & Symmetric stress in reference frame \\[4pt]
Cauchy Stress & $\displaystyle \boldsymbol{\sigma} = \frac{1}{J}\,\mathbf{F}\mathbf{S}\mathbf{F}^T$ & True stress in current frame \\[4pt]
Conjugate Pair & $\displaystyle \mathbf{S} : \dot{\mathbf{E}} = \dot{W}$ & Work rate per reference volume \\[4pt]
Small Strain Limit & $\displaystyle \sigma = E\,\varepsilon$ & Hooke’s law \\
\bottomrule
\end{tabular}
\end{center}
\endgroup

\subsection{The Effect of Material Symmetry}

---

\subsubsection{Concept}

Let two configurations \( k \) and \( r \) exhibit the same material properties.  
If a rotation \( \mathbf{Q} \) belongs to the \textbf{material symmetry group} of configuration \( k \),  
then the material in configuration \( r \) must exhibit identical constitutive behavior as in \( k \), regardless of the reference configuration.

This means that the constitutive function
\[
\mathbf{t} = \mathbf{f}(\mathbf{F})
\]
must satisfy
\[
\boxed{
\mathbf{f}(\mathbf{FQ}) = \mathbf{f}(\mathbf{F})
}
\quad \text{or equivalently} \quad
W(\mathbf{F}) = W(\mathbf{FQ})
\]
for all orthogonal \( \mathbf{Q} \in \mathcal{G} \), where \( \mathcal{G} \) is the material symmetry group.

---

\subsubsection{Orthogonal Members of the Symmetry Group}

Since \( \mathbf{Q}^T\mathbf{Q} = \mathbf{I} \), we require that:
\[
W(\mathbf{F}) = W(\mathbf{FQ}) = W(\mathbf{QF})
\]
and for the right Cauchy–Green tensor \( \mathbf{C} = \mathbf{F}^T\mathbf{F} \),
\[
(\mathbf{FQ})^T(\mathbf{FQ}) = \mathbf{Q}^T \mathbf{C} \mathbf{Q}
\]
Thus,
\[
\boxed{
W(\mathbf{C}) = W(\mathbf{Q}^T \mathbf{C} \mathbf{Q})
}
\]
Material objectivity (frame indifference) and material symmetry together imply that  
the strain energy must depend on \(\mathbf{C}\) through its invariants.

---

\subsection{Material Symmetry in the Infinitesimal Limit}

For infinitesimal strains \(\boldsymbol{\varepsilon}\),
\[
W(\boldsymbol{\varepsilon}) = W(\mathbf{Q}^T \boldsymbol{\varepsilon} \mathbf{Q})
\]
The **linear elastic solid** is defined by a quadratic strain energy density:
\[
\boxed{
W = \tfrac{1}{2} C_{ijkl}\, \varepsilon_{ij}\, \varepsilon_{kl}
}
\]
where \(C_{ijkl}\) are the **elastic stiffness constants**.

If \( \mathbf{Q} \) is a member of the material symmetry group, then so is \( \mathbf{Q}^T \),  
and the strain energy satisfies:
\[
\boxed{
W(\boldsymbol{\varepsilon}) = W(\mathbf{Q}\boldsymbol{\varepsilon}\mathbf{Q}^T)
}
\]

---

\subsection{Transformation of Elastic Constants}

Let \(\{\mathbf{e}_i\}\) and \(\{\mathbf{e}'_i\}\) denote the unrotated and rotated bases.  
If \(\mathbf{e}'_i = \mathbf{Q}\mathbf{e}_i\), then:
\[
\varepsilon'_{ij} = Q_{ip} Q_{jq} \varepsilon_{pq}
\]
and
\[
W(\boldsymbol{\varepsilon}') = W(\mathbf{Q}\boldsymbol{\varepsilon}\mathbf{Q}^T)
\]

For the linear hyperelastic solid:
\[
W(\boldsymbol{\varepsilon}) = \tfrac{1}{2} C_{ijkl}\, \varepsilon_{ij}\, \varepsilon_{kl}
\]
then
\[
W(\boldsymbol{\varepsilon}') = \tfrac{1}{2} C'_{pqmn}\, \varepsilon'_{pq}\, \varepsilon'_{mn}
= \tfrac{1}{2} C_{ijkl}\, \varepsilon_{ij}\, \varepsilon_{kl}
\]

Substituting the transformations of strain:
\[
C_{pqmn}\, \varepsilon_{pq}\, \varepsilon_{mn}
= Q_{pi}Q_{qj}Q_{mk}Q_{nl}\, C_{ijkl}\, \varepsilon_{pq}\, \varepsilon_{mn}
\]
and since this must hold for arbitrary strains:
\[
\boxed{
C_{pqmn} = Q_{pi} Q_{qj} Q_{mk} Q_{nl}\, C_{ijkl}
}
\]
Hence \( \mathbf{Q} \) is a **symmetry transformation** of the material if:
\[
\boxed{
C_{pqmn} = C_{ijkl}
}
\]

---

\subsection{Symmetries of \(C_{ijkl}\)}

Because \(W\) is a scalar quadratic form:
\[
W = \tfrac{1}{2} C_{ijkl}\, \varepsilon_{ij}\, \varepsilon_{kl}
\]
and since \(\varepsilon_{ij} = \varepsilon_{ji}\),  
we may exchange indices without loss of generality.

1. **Minor Symmetries:**
   \[
   C_{ijkl} = C_{jikl} = C_{ijlk}
   \]
   — symmetry with respect to interchanging the first or last pair of indices.

2. **Major Symmetry:**
   \[
   C_{ijkl} = C_{klij}
   \]
   — implied by the existence of strain energy \(W\).

Together, these symmetries reduce \(C_{ijkl}\) from 81 components to:
\[
\boxed{
36\ \text{independent constants from minor symmetry, and 21 from major symmetry.}
}
\]

---

\subsection{Linear Hyperelastic Material}

A linear hyperelastic (or linear elastic) material is one whose strain energy is quadratic in strain:
\[
W = \tfrac{1}{2} C_{ijkl}\, \varepsilon_{ij}\, \varepsilon_{kl},
\quad
\sigma_{ij} = \frac{\partial W}{\partial \varepsilon_{ij}} = C_{ijkl}\, \varepsilon_{kl}.
\]
This yields the **generalized Hooke’s law**:
\[
\boxed{
\sigma_{ij} = C_{ijkl}\, \varepsilon_{kl}
}
\]
The symmetry of stress and strain ensures that \(C_{ijkl} = C_{jikl} = C_{ijlk} = C_{klij}\),
reducing the number of independent constants to 21.

---

\subsection{Summary Table}

\begingroup
\setlength{\tabcolsep}{8pt}
\renewcommand{\arraystretch}{1.15}
\begin{center}
\begin{tabular}{@{} l l l @{}}
\toprule
\textbf{Symmetry Type} & \textbf{Expression} & \textbf{Effect on Constants} \\
\midrule
Minor symmetry (stress/strain symmetry) & 
$\displaystyle C_{ijkl} = C_{jikl} = C_{ijlk}$ & 
$81 \rightarrow 36$ \\[4pt]
Major symmetry (existence of $W$) & 
$\displaystyle C_{ijkl} = C_{klij}$ & 
$36 \rightarrow 21$ \\[4pt]
Material symmetry (e.g.\ isotropy) & 
$\displaystyle W(\mathbf{Q}\varepsilon\mathbf{Q}^T) = W(\varepsilon)$ & 
Further reduces constants \\[4pt]
Linear hyperelastic law & 
$\displaystyle \sigma_{ij} = C_{ijkl}\,\varepsilon_{kl}$ & 
Generalized Hooke’s law \\
\bottomrule
\end{tabular}
\end{center}
\endgroup

% \subsection{Matrix Notation and Elastic Symmetry}

% ---

% \subsubsection{Matrix (Voigt) Notation}

% Stress and strain tensors are written with a single index running from 1 to 6:
% \[
% \begin{array}{c|c}
% \text{Tensor Components} & \text{Matrix Form (Voigt)} \\ \hline
% \sigma_{11} = \sigma_1 & \varepsilon_{11} = \varepsilon_1 \\
% \sigma_{22} = \sigma_2 & \varepsilon_{22} = \varepsilon_2 \\
% \sigma_{33} = \sigma_3 & \varepsilon_{33} = \varepsilon_3 \\
% \sigma_{23} = \sigma_4 & \varepsilon_{23} = \varepsilon_4 \\
% \sigma_{31} = \sigma_5 & \varepsilon_{31} = \varepsilon_5 \\
% \sigma_{12} = \sigma_6 & \varepsilon_{12} = \varepsilon_6
% \end{array}
% \]

% The stress–strain relationship
% \[
% \sigma_{ij} = C_{ijkl}\,\varepsilon_{kl}
% \]
% becomes, in matrix form,
% \[
% \boxed{
% \sigma_m = C_{mn}\, \varepsilon_n, \quad m,n = 1,\dots,6.
% }
% \]

% In this notation, the pairs of tensor indices are replaced according to:
% \[
% (ij) = (11,22,33,23,31,12) \quad \Rightarrow \quad m = 1,2,3,4,5,6.
% \]

% For example:
% \[
% C_{1123} \;\to\; C_{14}, \qquad
% C_{3311} \;\to\; C_{31} = C_{13}.
% \]

% Thus:
% \[
% \sigma_i = C_{ij}\varepsilon_j
% \quad \text{or} \quad
% \sigma_1 = C_{11}\varepsilon_1 + C_{12}\varepsilon_2 + \cdots
% \]

% ---

% \subsubsection{Plane of Elastic Symmetry}

% If the material has one plane of elastic symmetry (e.g. about the $x_3$-axis),  
% the stiffness matrix takes the form:
% \[
% [C] =
% \begin{bmatrix}
% C_{11} & C_{12} & C_{13} & 0 & 0 & C_{16} \\
% C_{12} & C_{22} & C_{23} & 0 & 0 & C_{26} \\
% C_{13} & C_{23} & C_{33} & 0 & 0 & C_{36} \\
% 0 & 0 & 0 & C_{44} & C_{45} & 0 \\
% 0 & 0 & 0 & C_{45} & C_{55} & 0 \\
% C_{16} & C_{26} & C_{36} & 0 & 0 & C_{66}
% \end{bmatrix}
% \]
% This structure corresponds to **13 independent elastic constants.**

% ---

% \subsubsection{Orthotropic Symmetry}

% For orthotropic materials — having **three orthogonal planes of symmetry** —  
% the stiffness matrix reduces to:
% \[
% [C] =
% \begin{bmatrix}
% C_{11} & C_{12} & C_{13} & 0 & 0 & 0 \\
% C_{12} & C_{22} & C_{23} & 0 & 0 & 0 \\
% C_{13} & C_{23} & C_{33} & 0 & 0 & 0 \\
% 0 & 0 & 0 & C_{44} & 0 & 0 \\
% 0 & 0 & 0 & 0 & C_{55} & 0 \\
% 0 & 0 & 0 & 0 & 0 & C_{66}
% \end{bmatrix}
% \]
% This case has **9 independent elastic constants**  
% (typical for wood, fiber composites, rolled metal sheets, etc.).

% ---

% \subsubsection{Cubic Symmetry}

% For cubic materials (orthotropic + 90° rotational symmetry),
% \[
% [C] =
% \begin{bmatrix}
% C_{11} & C_{12} & C_{12} & 0 & 0 & 0 \\
% C_{12} & C_{11} & C_{12} & 0 & 0 & 0 \\
% C_{12} & C_{12} & C_{11} & 0 & 0 & 0 \\
% 0 & 0 & 0 & C_{44} & 0 & 0 \\
% 0 & 0 & 0 & 0 & C_{44} & 0 \\
% 0 & 0 & 0 & 0 & 0 & C_{44}
% \end{bmatrix}
% \]
% Cubic symmetry reduces the constants to **3 independent moduli**:  
% \(C_{11}\), \(C_{12}\), and \(C_{44}\).

% ---

% \subsubsection{Isotropic Symmetry}

% In isotropic materials, the stiffness matrix depends only on two constants:
% \[
% [C] =
% \begin{bmatrix}
% C_{11} & C_{12} & C_{12} & 0 & 0 & 0 \\
% C_{12} & C_{11} & C_{12} & 0 & 0 & 0 \\
% C_{12} & C_{12} & C_{11} & 0 & 0 & 0 \\
% 0 & 0 & 0 & \tfrac{1}{2}(C_{11}-C_{12}) & 0 & 0 \\
% 0 & 0 & 0 & 0 & \tfrac{1}{2}(C_{11}-C_{12}) & 0 \\
% 0 & 0 & 0 & 0 & 0 & \tfrac{1}{2}(C_{11}-C_{12})
% \end{bmatrix}
% \]

% If we define the **Lamé constants**:
% \[
% C_{11} = \lambda + 2\mu, \qquad C_{12} = \lambda,
% \]
% then
% \[
% [C] =
% \begin{bmatrix}
% \lambda+2\mu & \lambda & \lambda & 0 & 0 & 0 \\
% \lambda & \lambda+2\mu & \lambda & 0 & 0 & 0 \\
% \lambda & \lambda & \lambda+2\mu & 0 & 0 & 0 \\
% 0 & 0 & 0 & \mu & 0 & 0 \\
% 0 & 0 & 0 & 0 & \mu & 0 \\
% 0 & 0 & 0 & 0 & 0 & \mu
% \end{bmatrix}
% \]
% Thus, **isotropic solids** are completely described by only two independent elastic constants,  
% the Lamé parameters \( \lambda \) and \( \mu \).

% ---

% \subsection{Summary of Elastic Symmetry Classes}

% \begingroup
% \setlength{\tabcolsep}{8pt}
% \renewcommand{\arraystretch}{1.15}
% \begin{center}
% \begin{tabular}{@{} l c l @{}}
% \toprule
% \textbf{Symmetry Type} & \textbf{Independent Constants} & \textbf{Typical Examples} \\
% \midrule
% Triclinic (anisotropic) & 21 & General crystal \\[4pt]
% Monoclinic & 13 & Single-plane symmetric composites \\[4pt]
% Orthotropic & 9 & Wood, composites, rolled metals \\[4pt]
% Cubic & 3 & Metals (Al, Cu, Fe) \\[4pt]
% Isotropic & 2 & Rubber, glass, polycrystals \\
% \bottomrule
% \end{tabular}
% \end{center}
% \endgroup

% \subsection{Generalized Hooke’s Law (Isotropic Form)}

% \[
% \sigma_{ij} = \lambda\,\delta_{ij}\,\varepsilon_{kk} + 2\mu\,\varepsilon_{ij}
% \]
% \[
% E = \frac{\mu(3\lambda + 2\mu)}{\lambda + \mu},
% \quad
% \nu = \frac{\lambda}{2(\lambda + \mu)}.
% \]

\section*{Material Symmetry and Isotropic Elasticity}

\subsection{Matrix Notation and Elastic Symmetry}

\subsubsection{Matrix (Voigt) Notation}

Stress and strain tensors are written with a single index running from 1 to 6:
\[
\begin{array}{c|c}
\text{Tensor Components} & \text{Matrix Form (Voigt)} \\ \hline
\sigma_{11} = \sigma_1 & \varepsilon_{11} = \varepsilon_1 \\
\sigma_{22} = \sigma_2 & \varepsilon_{22} = \varepsilon_2 \\
\sigma_{33} = \sigma_3 & \varepsilon_{33} = \varepsilon_3 \\
\sigma_{23} = \sigma_4 & \varepsilon_{23} = \varepsilon_4 \\
\sigma_{31} = \sigma_5 & \varepsilon_{31} = \varepsilon_5 \\
\sigma_{12} = \sigma_6 & \varepsilon_{12} = \varepsilon_6
\end{array}
\]

The tensor relation
\[
\sigma_{ij} = C_{ijkl}\varepsilon_{kl}
\]
becomes, in matrix form,
\[
\sigma_m = C_{mn}\varepsilon_n, \qquad m,n = 1,\dots,6.
\]

Equivalently in vector form:
\begin{align}
\bm{\sigma} &=
\begin{bmatrix}
\sigma_{11} & \sigma_{22} & \sigma_{33} & \sigma_{23} & \sigma_{31} & \sigma_{12}
\end{bmatrix}^{\mathsf{T}},\\
\bm{\varepsilon} &=
\begin{bmatrix}
\varepsilon_{11} & \varepsilon_{22} & \varepsilon_{33} & 2\varepsilon_{23} & 2\varepsilon_{31} & 2\varepsilon_{12}
\end{bmatrix}^{\mathsf{T}}.
\end{align}

Thus the constitutive law becomes
\[
\bm{\sigma} = \mathbf{C}\,\bm{\varepsilon}.
\]

---

\subsubsection{Plane of Elastic Symmetry (Monoclinic)}

If the material has one plane of elastic symmetry (e.g.\ symmetry about the $x_3$-axis),  
the stiffness matrix takes the form:
\[
[C] =
\begin{bmatrix}
C_{11} & C_{12} & C_{13} & 0 & 0 & C_{16} \\
C_{12} & C_{22} & C_{23} & 0 & 0 & C_{26} \\
C_{13} & C_{23} & C_{33} & 0 & 0 & C_{36} \\
0 & 0 & 0 & C_{44} & C_{45} & 0 \\
0 & 0 & 0 & C_{45} & C_{55} & 0 \\
C_{16} & C_{26} & C_{36} & 0 & 0 & C_{66}
\end{bmatrix}
\]
This structure corresponds to **13 independent elastic constants.**

The principal axes of stress generally do *not* coincide with principal axes of strain, even if shear stresses vanish.

---

\subsubsection{Orthotropic Symmetry}

Materials with three orthogonal planes of symmetry have:
\[
[C] =
\begin{bmatrix}
C_{11} & C_{12} & C_{13} & 0 & 0 & 0 \\
C_{12} & C_{22} & C_{23} & 0 & 0 & 0 \\
C_{13} & C_{23} & C_{33} & 0 & 0 & 0 \\
0 & 0 & 0 & C_{44} & 0 & 0 \\
0 & 0 & 0 & 0 & C_{55} & 0 \\
0 & 0 & 0 & 0 & 0 & C_{66}
\end{bmatrix}
\]

This case has **9 independent constants.**

---

\subsubsection{Cubic Symmetry}

Imposing 90° rotational symmetry:
\[
[C] =
\begin{bmatrix}
C_{11} & C_{12} & C_{12} & 0 & 0 & 0 \\
C_{12} & C_{11} & C_{12} & 0 & 0 & 0 \\
C_{12} & C_{12} & C_{11} & 0 & 0 & 0 \\
0 & 0 & 0 & C_{44} & 0 & 0 \\
0 & 0 & 0 & 0 & C_{44} & 0 \\
0 & 0 & 0 & 0 & 0 & C_{44}
\end{bmatrix}
\]

Only **3 independent moduli** remain:  
\(C_{11},\, C_{12},\, C_{44}\).

---

\subsubsection{Isotropic Symmetry}

An isotropic solid has infinitely many planes of symmetry.  
The stiffness matrix reduces to:
\[
[C] =
\begin{bmatrix}
\lambda+2\mu & \lambda & \lambda & 0 & 0 & 0 \\
\lambda & \lambda+2\mu & \lambda & 0 & 0 & 0 \\
\lambda & \lambda & \lambda+2\mu & 0 & 0 & 0 \\
0 & 0 & 0 & \mu & 0 & 0 \\
0 & 0 & 0 & 0 & \mu & 0 \\
0 & 0 & 0 & 0 & 0 & \mu
\end{bmatrix}
\]

Thus isotropy is fully described by the **two Lamé constants** \( \lambda \) and \( \mu \).

---

\subsection{Summary of Elastic Symmetry Classes}

\begingroup
\setlength{\tabcolsep}{8pt}
\renewcommand{\arraystretch}{1.15}
\begin{center}
\begin{tabular}{@{} l c l @{}}
\toprule
\textbf{Symmetry Type} & \textbf{Independent Constants} & \textbf{Typical Examples} \\
\midrule
Triclinic (anisotropic) & 21 & General crystal \\
Monoclinic & 13 & Laminated composites \\
Orthotropic & 9 & Wood, composites, rolled metals \\
Cubic & 3 & Polycrystalline metals \\
Isotropic & 2 & Rubber, glass, polycrystals \\
\bottomrule
\end{tabular}
\end{center}
\endgroup

---

\subsection{Isotropic Fourth-Order Tensor}

The general isotropic elasticity tensor is:
\[
C_{ijkl}
= \lambda\,\delta_{ij}\delta_{kl}
+ \mu\left(\delta_{ik}\delta_{jl} + \delta_{il}\delta_{jk}\right),
\]
with the constitutive relation:
\[
\sigma_{ij} = \lambda \varepsilon_{kk}\delta_{ij} + 2\mu \varepsilon_{ij}.
\]

---

\subsection{Lamé Constants and Engineering Moduli}

\[
\lambda = \text{first Lamé parameter}, \qquad 
\mu = \text{shear modulus}.
\]

\[
K = \lambda + \frac{2}{3}\mu, \qquad
E = \frac{\mu(3\lambda + 2\mu)}{\lambda + \mu}, \qquad
\nu = \frac{\lambda}{2(\lambda + \mu)}.
\]

Inverted:
\[
\mu = \frac{E}{2(1+\nu)}, \qquad
K = \frac{E}{3(1-2\nu)}.
\]

---

\subsection{Uniaxial Loading and Incompressibility}

Given
\[
\sigma_{11}=\sigma,\qquad \sigma_{22}=\sigma_{33}=0,
\]
the strains are:
\[
\varepsilon_{11} = \frac{\sigma}{E}, \qquad
\varepsilon_{22}=\varepsilon_{33} = -\nu\frac{\sigma}{E}.
\]

Incompressibility requires:
\[
\varepsilon_{kk}=0 \;\Rightarrow\; \nu \to \tfrac{1}{2}.
\]

---

\subsection{Wave Speed and Elastic Modulus (Sonic Test)}

Longitudinal wave speed in an isotropic rod:
\[
c_L = \sqrt{\frac{E(1-\nu)}{\rho(1+\nu)(1-2\nu)}}.
\]

Useful for extracting \(E\) experimentally from measured \(c_L\).

---
\subsection{Isotropic Linear Elasticity}

---

\subsubsection{Isotropic Material Tensor Form}

For an isotropic material, the stiffness tensor \( C_{ijkl} \) satisfies:
\[
C_{ijkl} = Q_{ip} Q_{jq} Q_{kr} Q_{ls} C_{pqrs} \quad \forall Q \in SO(3)
\]
This implies \( C_{ijkl} \) must be an **isotropic fourth-order tensor**, expressible as:
\[
\boxed{
C_{ijkl} = \lambda \, \delta_{ij}\delta_{kl} + \mu \, (\delta_{ik}\delta_{jl} + \delta_{il}\delta_{jk})
}
\]

Thus, the **constitutive law** (Hooke’s law) becomes:
\[
\sigma_{ij} = C_{ijkl} \, \varepsilon_{kl} = \lambda \, \varepsilon_{kk}\delta_{ij} + 2\mu \, \varepsilon_{ij}
\]

---

\subsubsection{Simplified Form and Interpretation}

\[
\boxed{
\sigma_{ij} = \lambda (\nabla \cdot \mathbf{u})\delta_{ij} + 2\mu \varepsilon_{ij}
}
\]
where \( \varepsilon_{ij} = \tfrac{1}{2}(u_{i,j} + u_{j,i}) \) is the infinitesimal strain tensor.

For hydrostatic stress,
\[
\sigma_{m} = \frac{1}{3}(\sigma_{11} + \sigma_{22} + \sigma_{33})
\quad \Rightarrow \quad
\frac{\sigma_{m}}{3} = K \varepsilon_{m}
\]
where \( K \) is the **bulk modulus**:
\[
\boxed{
K = \lambda + \tfrac{2}{3}\mu
}
\]

The pressure \( p = -\tfrac{1}{3}\sigma_{kk} \) is related to volumetric strain:
\[
p = -K \, \frac{\Delta V}{V}
\]

---

\subsubsection{Axial Strain Relations}

For uniaxial tension in the \(x_1\)-direction:
\[
\varepsilon_{11} = \frac{\sigma_{11}}{E}, \qquad
\varepsilon_{22} = \varepsilon_{33} = -\nu \frac{\sigma_{11}}{E}
\]
where:
\[
E = \text{Young’s modulus}, \quad \nu = \text{Poisson’s ratio}.
\]

---

\subsubsection{General 3D Relations}

For principal stresses \(\sigma_1, \sigma_2, \sigma_3\):
\[
\boxed{
\begin{aligned}
\varepsilon_1 &= \frac{1}{E}\left[\sigma_1 - \nu(\sigma_2 + \sigma_3)\right], \\
\varepsilon_2 &= \frac{1}{E}\left[\sigma_2 - \nu(\sigma_3 + \sigma_1)\right], \\
\varepsilon_3 &= \frac{1}{E}\left[\sigma_3 - \nu(\sigma_1 + \sigma_2)\right].
\end{aligned}
}
\]
For hydrostatic stress:
\[
\varepsilon_{ij} = \frac{1+\nu}{E}\sigma_{ij} - \frac{\nu}{E}\sigma_{kk}\delta_{ij}
\]
and equivalently,
\[
\sigma_{ij} = \frac{E}{(1+\nu)(1-2\nu)}\left[(1-\nu)\varepsilon_{ij} + \nu\varepsilon_{kk}\delta_{ij}\right].
\]

---

\subsubsection{Relations Between Elastic Moduli}

All isotropic moduli are related as:
\[
\boxed{
\begin{aligned}
E &= \frac{\mu(3\lambda + 2\mu)}{\lambda + \mu}, \\
\nu &= \frac{\lambda}{2(\lambda + \mu)}, \\
K &= \lambda + \tfrac{2}{3}\mu, \\
\mu &= \frac{E}{2(1+\nu)}, \\
K &= \frac{E}{3(1-2\nu)}.
\end{aligned}
}
\]
Only two of these are **independent**.

---

\subsubsection{Incompressible Limit}

For incompressible materials:
\[
\varepsilon_{11} + \varepsilon_{22} + \varepsilon_{33} = 0,
\quad \nu = \tfrac{1}{2},
\quad K \to \infty.
\]
Then:
\[
\sigma_{ij} = -p\,\delta_{ij} + 2\mu \varepsilon_{ij}
\]
where \(p\) acts as a **Lagrange multiplier** enforcing incompressibility.

---

\subsubsection{Boundary Value Problem (BVP)}

For infinitesimal linear elasticity:

\[
\begin{cases}
\sigma_{ij,j} + b_i = 0 & \text{in } V, \\
\sigma_{ij} = 2\mu \varepsilon_{ij} + \lambda \varepsilon_{kk}\delta_{ij}, \\
\varepsilon_{ij} = \tfrac{1}{2}(u_{i,j} + u_{j,i}),
\end{cases}
\]

with boundary conditions:
\[
\begin{cases}
u_i = \bar{u}_i & \text{on } S_u \quad (\text{displacement BC}),\\
t_i = \sigma_{ij}n_j = \bar{t}_i & \text{on } S_t \quad (\text{traction BC}),
\end{cases}
\]
where \(S = S_u \cup S_t\).

This gives:
- \(15\) unknowns: \(u_i, \varepsilon_{ij}, \sigma_{ij}\),
- \(15\) equations: \(6\) kinematic + \(6\) constitutive + \(3\) equilibrium.

\[
\boxed{
\text{Total BVP: } 
\begin{cases}
\sigma_{ij,j} + b_i = 0, \\
\sigma_{ij} = \lambda \varepsilon_{kk}\delta_{ij} + 2\mu \varepsilon_{ij}, \\
\varepsilon_{ij} = \tfrac{1}{2}(u_{i,j} + u_{j,i}).
\end{cases}
}
\]

---

\subsection{Summary}

\begingroup
\setlength{\tabcolsep}{8pt}
\renewcommand{\arraystretch}{1.15}
\begin{center}
\begin{tabular}{@{} l l l @{}}
\toprule
\textbf{Symbol} & \textbf{Definition} & \textbf{Name} \\
\midrule
$E$ & Young’s modulus & Axial stiffness \\[4pt]
$\nu$ & Poisson’s ratio & Lateral contraction \\[4pt]
$\lambda$ & 1st Lamé constant & Volumetric response \\[4pt]
$\mu$ & 2nd Lamé constant & Shear modulus \\[4pt]
$K$ & $\displaystyle \lambda + \frac{2}{3}\mu$ & Bulk modulus \\
\bottomrule
\end{tabular}
\end{center}
\endgroup

\noindent\textbf{Note:} Only two of these constants are independent for an isotropic solid.---

\subsection{Boundary Value Problems in Linear Elasticity}

---

\subsubsection{Types of Boundary Conditions}

\begin{itemize}
\item \textbf{Displacement BVP:} 
  \[
  S_u = S, \quad S_t = 0
  \]
  Displacements \( u_i \) are prescribed on the entire boundary.

\item \textbf{Traction BVP:}
  \[
  S_t = S, \quad S_u = 0
  \]
  Surface tractions \( T_i \) are prescribed on the entire boundary.

\item \textbf{Mixed BVP:}
  At each boundary point, specify either the displacement or the traction (but not both).  
  For example:
  \[
  \text{At one point: } T_1,\, u_2,\, u_3 \quad \text{and at another: } u_1,\, T_2,\, T_3.
  \]
\end{itemize}

---

\subsection{Example: Uniform Pressure on a 3D Body}

Consider a traction BVP under a uniform hydrostatic pressure \( P \).

\[
T_i^{(S)} = -P n_i
\]

\[
\text{Guess: } \sigma_{ij} = -P \delta_{ij}
\quad \Rightarrow \quad
[\sigma] =
\begin{bmatrix}
-P & 0 & 0 \\
0 & -P & 0 \\
0 & 0 & -P
\end{bmatrix}
\]

---

\subsubsection{Boundary Conditions}

Cauchy’s relationship:
\[
T_i = \sigma_{ij} n_j = -P n_i
\]
is satisfied.

---

\subsubsection{Equilibrium Equation}

\[
\sigma_{ij,j} + b_i = 0
\]
\[
(-P \delta_{ij})_{,j} = 0 \quad \Rightarrow \quad 0 = 0 \quad \checkmark
\]

---

\subsubsection{Constitutive Equations}

For isotropic linear elasticity:
\[
\varepsilon_{ij} = \frac{1+\nu}{E}\sigma_{ij} - \frac{\nu}{E}\sigma_{kk}\delta_{ij}
\]
Substitute \(\sigma_{ij} = -P \delta_{ij}\):
\[
\varepsilon_{ij} = \frac{1+\nu}{E}(-P \delta_{ij}) - \frac{\nu}{E}(-3P)\delta_{ij}
= -\frac{P}{3K}\delta_{ij}
\]
with \( K = \frac{E}{3(1-2\nu)} \) the **bulk modulus**.

Thus:
\[
\boxed{
[\varepsilon] =
\begin{bmatrix}
-\frac{P}{3K} & 0 & 0 \\
0 & -\frac{P}{3K} & 0 \\
0 & 0 & -\frac{P}{3K}
\end{bmatrix}
}
\]

---

\subsubsection{Compatibility}

Yes — since the compatibility equations involve only second derivatives of \(\varepsilon_{ij}\),  
and \(\varepsilon_{ij}\) is constant, they are automatically satisfied.

---

\subsubsection{Displacements}

From \(\varepsilon_{ii} = \partial u_i / \partial x_i = \alpha\):
\[
\alpha = -\frac{P}{3K}
\]
Integrating:
\[
u_i = \alpha x_i + A_i(x_j,x_k)
\]
and requiring no shear strains:
\[
2\varepsilon_{ij} = u_{i,j} + u_{j,i} = 0
\]
implies \( A_i \) are constants or linear in coordinates,  
corresponding to **rigid body translations and rotations**.

Hence:
\[
\boxed{
u_i = -\frac{P}{3K}x_i + \text{(rigid motion)}
}
\]

---

\subsubsection{6. Volume Change}

\[
\Delta V = \int_V \varepsilon_{kk}\, dV
= 3\left(-\frac{P}{3K}\right) V
= -\frac{P}{K}V
\]

Thus, the **fractional volume change** is:
\[
\boxed{
\frac{\Delta V}{V} = -\frac{P}{K}
}
\]
consistent with the definition of the bulk modulus.

---

\subsection{Uniqueness of Solution}

The solution is **unique** because the strain energy density is **positive definite**:
\[
W = \tfrac{1}{2} C_{ijkl}\varepsilon_{ij}\varepsilon_{kl} > 0, \quad \forall \varepsilon_{ij} \neq 0.
\]
For isotropic elasticity, this implies:
\[
\mu > 0, \qquad 3\lambda + 2\mu > 0.
\]

In terms of \(E\) and \(\nu\):
\[
E > 0, \quad -1 < \nu < \tfrac{1}{2}.
\]
If \(E \le 0\) or \(\nu \notin (-1, \tfrac{1}{2})\), the strain energy would not be positive definite.

---


---

\subsection{Summary}

\begingroup
\setlength{\tabcolsep}{8pt}
\renewcommand{\arraystretch}{1.15}
\begin{center}
\begin{tabular}{@{} l l l @{}}
\toprule
\textbf{Concept} & \textbf{Expression} & \textbf{Notes} \\
\midrule
Constitutive law & 
$\displaystyle \sigma_{ij} = \lambda \varepsilon_{kk}\delta_{ij} + 2\mu \varepsilon_{ij}$ & 
Linear isotropic elasticity \\[4pt]

Bulk modulus & 
$\displaystyle K = \lambda + \frac{2}{3}\mu$ & 
Relates hydrostatic stress to volumetric strain \\[4pt]

Hydrostatic solution & 
$\displaystyle \sigma_{ij} = -P\delta_{ij}, \quad \varepsilon_{ij} = -\frac{P}{3K}\delta_{ij}$ & 
Uniform compression \\[4pt]

Volume change & 
$\displaystyle \frac{\Delta V}{V} = -\frac{P}{K}$ & 
Consistent with definition of $K$ \\[4pt]

Uniqueness & 
$\displaystyle E>0, \quad -1<\nu<\tfrac{1}{2}$ & 
Positive-definiteness condition \\
\bottomrule
\end{tabular}
\end{center}
\endgroup
\end{document}
